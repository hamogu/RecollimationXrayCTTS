\subsubsection{Initial wind temperature}
\label{sect:T_0}
In Section~\ref{sect:model} we assume an initially cold stellar wind where the thermal pressure before the shock can be ignored. This is motivated by two arguments: (i) \citet{2007IAUS..243..299M} show that {\it hot} stellar winds will cool quickly and cause X-ray emission much brighter than observed if they have a mass loss rate above $10^{-11}M_\odot\mathrm{ yr}^{-1}$. Thus, only stellar winds that are launched {\it cold} can provide the required mass loss rates well above this value (see Section~\ref{sect:masslossrates}). (ii) The solar wind has 1~MK. It is unclear which physical process heats it to this temperature, but it is probably related to magnetic waves. In CTTS the wind mass loss rate is \textbd{much} higher than in the Sun, and it seems \textbf{likely winds from CTTS are cooler than the solar wind, even if they are heated by the same process.}

The approximation to neglect the initial wind temperature is valid to a few $10^5$~K for the densities and wind speeds considered here. Hotter winds cannot be described in our model, because their sound speed is so large that no strong shock develops for small angles $\psi$.