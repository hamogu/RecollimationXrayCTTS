\subsection{Initial wind temperature}
\label{sect:T_0}
In Section~\ref{sect:model} we developed our model for an initially cold stellar wind where the thermal presure before the shock can be ignored. This the motivated by two arguments: (i) \citet{2007IAUS..243..299M} show that hot stellar winds will cool quickly and cause X-ray emission much bighter than observed if they have a mass loss rate above $10^{-11}M_\odot\mathrm{ yr}^{-1}$. Since we need mass loss rates in excess of this value to explain the resolved X-ray emission at 40~AU, the wind must be launched cold. (ii) The solar wind has 1~MK. It is not clear which physical process heats it to this temperature, but it is probably related to magnetic waves. In CTTS the wind mass loss rate is much higher than in the sun, but it is unknown if the same mechanims can provide enough energy to heat a stellar CTTS wind to the same temperature.

Nevertheless, we want to explore how different the shock front would look for an intially hot wind.
In this case eqn.~\ref{eqn:P} is
\begin{eqnarray}
P(z) & = & \rho_0 v_0^2 + \rho_0 \frac{k T_0}{\mu m_{\textrm{H}}} \nonumber\\ 
     & = & \frac{\dot{M}}{4\pi v_{\infty}(z^2+\omega^2)} \left( v_{\infty}^2 \sin^2(\psi) + \frac{kT}{\mu m_{\textrm{H}}}\right)\\
\end{eqnarray}
and thus eqn.~\ref{eqn:ode} becomes
\begin{equation}
\frac{\rm{d}\omega}{\rm{d}z} = \tan\left[\arctan\left(\frac{\omega}{z}\right)-\arcsin\sqrt{\frac{z^2+\omega^2}{R_0^2}-\frac{kT}{\mu m_{\textrm{H} v_{\infty}^2}}}\right]\ .
\end{equation}
The post-shock temperature $T_{\mathrm{post-shock}}$ is now the sum of the value calculated in eqn.~\ref{eqn:T} and $T_0$.

For small radii and large $T_0$ the term in the last square root can become negative. Physically, this corresponds to situations where the thermal pressure is much larger than $P(z)$. The shock front is pushed outwards. Without the thermal pressure $\psi$ is always positive, because the ram pressure approches 0 when the shock front is almost parallel to the flow direction of the wind. In a hot wind, there is no such limit and the thermal pressure can force the shock front below the plane of the disk. However, in CTTS the disk provides a solid barrier and, this close to the star, the magnetic field influences the direction of the flow. To explore what a hot wind would do to the shape of the shock front while avoiding this problem, we start the integration at $\omega=1$~AU. Figure~\ref{fig:T_0} shows that a wind with just $T_0=10^5$~K leads to essentially the same shape in the shock front as a cold wind, only the post-shock temperature distribution is slightly shifted (rightmost panel). Even for much higher temperatures around $T_0=10^6$~K the shape does not change dramatically. However, almost the entire emission measure of the wind is now shifted into the temperature regime where we expect X-ray radiation.
