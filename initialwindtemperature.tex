\subsubsection{Initial wind temperature}
\label{sect:T_0}
In Section~\ref{sect:model} we developed our model for an initially cold stellar wind where the thermal presure before the shock can be ignored. This is motivated by two arguments: (i) \citet{2007IAUS..243..299M} show that hot stellar winds will cool quickly and cause X-ray emission much brighter than observed if they have a mass loss rate above $10^{-11}M_\odot\mathrm{ yr}^{-1}$. Section~\ref{sect:masslossrates} shows that mass loss rates well above this value are required to explain the resolved X-ray emission at 40~AU, thus the wind must be launched cold. (ii) The solar wind has 1~MK. It is unclear which physical process heats it to this temperature, but it is probably related to magnetic waves. In CTTS the wind mass loss rate is higher than in the Sun, and it seems unlikly that the same mechanim can provide enough energy to heat a stellar CTTS wind to the same temperature.

The approximation to neglect the initial wind temperature is valid to a few $10^5$~K for the densities and wind speeds considered here. Hotter winds cannot be described in our model, because their sound speed is so large that no strong shock develops for small angles $\psi$.