\subsection{Disk winds as boundary conditions for stellar winds}
\label{sect:boundary}
Different models exist to explain wind launching from the stellar surface \citep{1988ApJ...332L..41K,2005ApJ...632L.135M}, the X-point close to the inner disk edge \citep{1994ApJ...429..781S} and magneto-centrifugal launching from the disk \citep{1982MNRAS.199..883B,2005ApJ...630..945A}. It is likely that more than one mechanism contributes to the total outflow from the system. In this case, we expect a contact discontinuity between the different components whose position is determined by the pressure on both sides. Specifically, hydromagnetic disk winds have a tendency to collimate and possibly even to recollimate to smaller flow radii under certain conditions \citep{1982MNRAS.199..883B,1992ApJ...394..117P}.
Numerically, the magneto-centrifugally accelerated disk wind is probably the best explored component. Magneto-hydrodynamic (MHD) simulations of the disk wind have been performed in 2D \citep[e.g.][]{2005ApJ...630..945A}, 2.5D \citep[e.g.][]{2011ApJ...728L..11R} or 3D \citep[e.g.][]{2006ApJ...653L..33A}, but typically do not resolve the stellar wind. However, they show that the disk wind is collimated close to the axis and that the densities are largest in this region. Furthermore, the Alfv\'en surface (which separates the magnetically dominated region from the flow-dominated region) is located at many AU for the inner layers of the jet. This is in contrast with the outer, less collimated layers of the wind, which leave the magnetically dominated region at a few AUs.

\citet{2009A&A...502..217M} present analytical and numerical solutions for several scenarios that mix an inner stellar wind and an outer disk wind \citep[this model has been extended in ][]{2012A&A...545A..53M,2014A&A...562A.117T}. In contrast to our approach, they impose a smooth transition between stellar wind and disk wind and they start their simulation at $z=50$~AU instead of at the star. With some time variability in the wind launching their models produce promising knot features in the jet. In the context of our analysis, we note that the pressure in their models is magnetically dominated and that Kompaneet's approximation does not hold in the disk wind, but that the presure gradient is small on scales of a few AU. Also, the inner jet component discussed here is smaller than the innermost resolution elements for the fiducial parameters. The pressure at the jet axis is high in the plane of the disk and drops by one to two orders of magnitude until it reaches a plateau at $P_\infty$. Below we use an exponential $P(z)=P_\infty+P_0\exp\left(-\frac{z}{h}\right)$ to mimic this profile.
Similar profiles for the inner density and pressure are seen in simulations by other groups \citep[e.g.][]{2005ApJ...630..945A,Li_Krasnopolsky_Blandford_2006,2008ApJ...678.1109M}.

Figure~\ref{fig:p_ext} shows how different pressure profiles influence the shock position. Larger pressures force the shock front onto the symmetry axes for smaller $z$ (top row). If the pressure is constant in the region where the shock front hits the symmetry axis, then the angle between the shock front and the jet axis is large, which causes high post-shock temperatures (solid red line in top row). On the other hand, if there is a pressure gradient when the shock front bends towards the jet axis, then the shock front and the stream lines form a smaller angle and the post-shock temperatures are lower (dotted black line in the top row).

The solutions shown in the bottom row of the figure are for the same pressure scale heights $h$ as those in the upper row, but here we use smaller $P_0$ for scenarios with large $h$, so that the shock front reaches the jet axis at approximately the same $z$. Close to the disk plane the pre-shock speeds differ significantly, but at large $z$ they reach very similar values. The scenarios with smaller $P_0$ reach larger radii and the slightly different shape of the shock front leads to more plasma at high temperatures.