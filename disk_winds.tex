\subsection{Disk winds as boundary conditions for stellar winds}
Different models exist to explain wind launching from the stellar surface \citep{1988ApJ...332L..41K,2005ApJ...632L.135M}, the X-point close to the inner disk edge \citep{1994ApJ...429..781S} and magneto-centrifugal launching from the disk \citep{1982MNRAS.199..883B,2005ApJ...630..945A}. It is likely that more than one mechanism contributes to the total outflow from the system. In this case, we expect a contact discontinuity between the different components. Numerically, the magneto-centrifugally accelerated disk wind is probably the best explored component. Magneto-hydrodynamic (MHD) simulations of the disk wind have been performed in 2D \citep[e.g.][]{2005ApJ...630..945A}, 2.5D \citep[e.g.][]{2011ApJ...728L..11R} or 3D \citep[e.g.][]{2006ApJ...653L..33A}, but typically do not resolve the stellar wind, where the magneto-centrifugal launching is not effective. However, they show that the disk wind is collimated close to the axis and that the densities are largest in this region. Furthermore, the inner layers of the outflow close to the jet axes are within the Alfv\'en surface for many AU, the boundary between a magnetically dominated flow and a gas-pressure dominated flow. This is in contrast to the outer, less collimated layers of the wind, which leave the magnetically dominated region at a few AU.

\citet{2009A&A...502..217M} present analytical and numerical solutions for several scenarios that mix an inner stellar wind and an outer disk wind. In contrast to our approach, they impose a smooth transition between stellar wind and disk wind, which allows them to model the entire outflow region numerically. With some time variability in the wind launching their models produce promising knot features in the jet. In the context of our analysis, we note that the pressure in their models is magnetically dominated and much higher close to the jet axis than at larger radii in apparent contrast to Kompaneet's approximation. However, the inner jet component, that we discuss here, is so narrow that it only feels the pressure in the innermost resolution elements. The pressure at the jet axis is high initially and drops by one to two orders of magnitude until it reaches a plateau at $P_\infty$. Below we use simple exponential ($P(z)=P_\infty+P_0\exp\left(-\frac{z}{h}\right)$) or power-law functions to mimic this profile.

Similar profiles for the inner density and pressure are seen in simulations by competing groups \citep[e.g.]{2005ApJ...630..945A,Li_Krasnopolsky_Blandford_2006,2008ApJ...678.1109M}.

Observations of jets and winds from CTTS indicate that typical temperatures are a few thousand K and typical densities are in the range $10^4-10^5 \mathrm{ cm}^{-3}$ \citep[e.g.][]{2000A&A...356L..41L,2007ApJ...657..897K}. We chose the parameters of $P(z)$ to reach pressures compatible with the observed densities and temperatures.

Figure~\ref{fig:p_ext} shows how different pressure profiles influence the shock position. 
Larger pressures force the shock front onto the symmetry axes for smaller $z$ (top row). If the pressure profile is almost linear in the region where the shock front hits the symmetry axis, then the angle between the shock front and the jet axis is large, which causes high post-shock temperatures (dotted black and solid red line in the upper row). In contrast, if the pressure gradient becomes less steep when the shock front bends towards the jet axis, then the shock front and the stream lines form a smaller angle and pre-shock speeds and thus the post-shock temperatures are lower (dashed green line in the top row).

The solutions shown in the bottom row of the figure are for the same scale heights as those in the upper row, but here we use smaller $P_0$ for scenarios with large scake heights $h$, so that the shock front reaches the jet axis at approximately the same $z$. Close to the disk plane the pre-shock speeds differ significantly, but at large $z$ they reach very similar values. However, the scenarios with the smaller $P_0$ values reach larger radii and the slightly different shape of the shock front leads to more plasma at high temperatures.