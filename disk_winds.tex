\subsection{Disk wind as boundary conditions for a stellar wind}
Different models exist to explain wind launching from the stellar surface \citep{1988ApJ...332L..41K,2005ApJ...632L.135M}, the X-point close to the inner disk edge \citep{1994ApJ...429..781S} and magneto-centrifugal launching from the disk \citep{1982MNRAS.199..883B,2005ApJ...630..945A}. It is likely that more than one mechanism contributes to the total outflow from the system. In this case, we expect a contact discontinuity between the different components. Numerically, the magneto-centrifugally accelerated disk wind is probably the best explored component. Magneto-hydrodynamic (MHD) simulations of the disk wind have been performed in 2D \citep{2005ApJ...630..945A}, 2.5D \citep{2011ApJ...728L..11R} or 3D \citep{2006ApJ...653L..33A}, but typically do not resolve the stellar wind, where the magneto-centrifugal launching is not effective. However, they show that the disk wind is collimated close to the axis and that the densities are largest in this region. Furthermore, the inner layers of the outflow close to the jet are within the Alfv\'en surface, the boundary between a magnetically dominated flow and a gas-pressure dominated flow, even at distances of several tens of AU from the central star, in contrast to the outer, less collimated layers of the wind, which leave the magnetically dominated region at a few AU.

\citet{2009A&A...502..217M} present analytical and numerical solutions for several scenarios that mix an inner, presumably stellar, wind and an outer disk wind. In contrast to our approach, they impose a smooth transition between stellar wind and disk wind, which allows them to model the entire outflow region numerically. With some time variability in the wind launching their models produce promising knot features in the jet. In the context of our analysis, we note that the presure in their models is magnetically dominated and much higher close to the jet axies than at larger radii in apparent contrast to Kompaneet's approximation. However, the an inner jet component as suggested in this article is so narrow that it essentiall stays confind to the innermost resolution elements. The presure at the jet axis is high initially and reaches a plateau after dropping by one to two orders of magnitude. Below we use simple exponential or power-law functions for P(z) that mimic this behaviour.

Similar profiles for the inner density and presure are seen in simulations by competing groups \citep[e.g.]{http://dx.doi.org/10.1086/432040,Li_Krasnopolsky_Blandford_2006,2008ApJ...678.1109M}.