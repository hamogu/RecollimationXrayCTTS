\section{Summary and conclusion}
\label{sect:conclusion}
In a a CTTS system stellar wind and disk wind interact. The high pressure of the disk wind can collimate the stellar wind into a directed jet. This collimation builds up a shock surface within the stellar wind and for reasonable parameters of $\dot M$, $v_\infty$, $\omega_0$ and $P(z)$ the shock surface encloses a region only few AU wide and a few tens of AU extended along the jet axis. 
Paper~I shows that the post-shock cooling zone behind the shock front is of similar size.
 
This is so small that it cannot be resolved with current instrumentation and therefore cannot be seen directly as cavity in the disk wind. A small fraction of the stellar wind is shocked to X-ray emitting temperatures $>1$~MK. 
Paper~I showed that a shock with these properties is required to explain the observed X-ray emission in the CTTS DG~Tau and here we now explain how these shock properties naturally arise in a scenario where the stellar wind is collimated by the external presure and form the innermost layers of the jet.
Furthermore, \citet{2013A&A...550L...1S} observed C\;{\sc iv} emission in DG Tau that is formed at cooler temperatures than the X-ray emission and that is too luminous to be explained by cooled X-ray plasma alone. Our solutions to the ODE here almost all have more plamsa heated to 0.5~MK, then to 1~MK and can thus explain these observations as well.

However, more detailed numerical simulations of the post-shock cooling zone and the shape of the contact discontinuity between the disk wind and the post-shock stellar wind are required to check weather the physical extend of the cooling region behind the shock front and the position of the peak emission can be matched to the observations at the same time.

There is too much parameter degeneracy to turn the argument around and to derive $\dot M$, $v_\infty$ or $P(z)$ from the fact that we observe X-ray and FUV emission a few tens of AU from the star.

In summary, a fest stellar wind that is collimated by external pressure from the disk wind will form a collimation shock. We derive the geometrical shape and other properties of this shock front and find that this model is a plausible scenario to explain soft X-ray and FUV emission as observed at the base of young stellar jets, specifically in DG~Tau.


