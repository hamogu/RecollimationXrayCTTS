\section{Summary and conclusion}
\label{sect:conclusion}
In a CTTS system stellar wind and disk wind interact. The high pressure of the disk wind can confine the free stellar wind into a narrow, jet-like region, bound by an elongated shock surface. For reasonable parameters of $\dot M$, $v_\infty$, $\omega_0$ and $P(z)$ the shock surface encloses a region only several AU wide but tens of AUs along the jet axis. 
Paper~I shows that the post-shock cooling zone behind this shock front is of similar size.
 
The shock surface is so small that it cannot be resolved with current instrumentation and therefore cannot be seen directly as a cavity in the disk wind. A small fraction of the stellar wind is shocked to X-ray emitting temperatures $>1$~MK. 
Paper~I showed that a shock with these properties is required to explain the observed X-ray emission in the CTTS DG~Tau. We show that such a shock naturally arises in a scenario where the stellar wind is confined by an external presure and feeds the innermost layers of the jet.
Furthermore, \citet{2013A&A...550L...1S} observed C\;{\sc iv} emission in DG Tau that is formed at cooler temperatures than those required for X-ray emission and that is too luminous to be explained by cooled X-ray plasma alone. Almost all solutions of the ODE describing the interaction of stellar and disk wind have more plamsa heated to 0.5~MK than to 1~MK and can in principle explain these observations as well.

However, more detailed numerical simulations of the post-shock cooling zone and the shape of the contact discontinuity between the disk wind and the post-shock stellar wind are required to check whether the physical extent of the cooling region behind the shock front and the position of the peak  C\;{\sc iv} emission can be matched to the observations at the same time.

There is too much parameter degeneracy to turn the argument around and derive $\dot M$, $v_\infty$ or $P(z)$ from the fact that we observe X-ray and FUV emission a few tens of AUs from the star.

In summary, a fast stellar wind that is confined by an external pressure from the disk wind will form a collimation shock. We derive the geometrical shape and other properties of this shock front and find that this model is a viable explanation of the soft X-ray and FUV emission observed at the base of young stellar jets, specifically in DG~Tau.

Acknowledgement: Support for program #____ was provided by NASA through a grant from the Space Telescope Science Institute, which is operated by the Association of Universities for Research in Astronomy, Inc., under NASA contract NAS 5-26555.
Support for this work was provided for HMG by NASA through grant GO-12907.01-A from the Space Telescope Science Institute, which is operated by the Association of Universities for Research in Astronomy, Inc., under NASA contract NAS 5-26555 and by NSF AST1313083 and NASA NNX14AB38G (ZYL).