\section{Summary}
\label{sect:summary}
A fast stellar wind that is confined by an external pressure from the disk wind will form a stationary collimation shock. We derive the geometrical shape and other properties of this shock front and find that this model provides a viable explanation of the soft X-ray and FUV emission observed at the base of young stellar jets, specifically in DG~Tau.

Acknowledgement: Support for this work was provided for HMG by NASA through grant GO-12907.01-A from the Space Telescope Science Institute, which is operated by the Association of Universities for Research in Astronomy, Inc., under NASA contract NAS 5-26555 and by NSF AST1313083 and NASA NNX14AB38G for ZYL.