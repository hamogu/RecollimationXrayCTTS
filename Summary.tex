\section{Summary and conclusion}
\label{sect:summary}
A fast stellar wind that is confined by an external pressure from the disk wind will form a stationary collimation shock \citep{2012MNRAS.422.2282K}. We derive the geometrical shape and other properties of this shock front and find that the shock front closes back the axis of symmetry and encloses a roughly egg-shaped volume for all combinations of parameters discussed in this article. This model provides a viable explanation of the soft X-ray and FUV emission observed at the base of young stellar jets. Specifically for DG~Tau, we fit the parameters of the stellar wind to the observed X-ray spectrum and the observed distance between star and X-ray emission.  We expect recollimation shocks to be present in every jet, but only in the case of DG~Tau it is possible to test this observationally. 

This analysis shows that collimation shocks in a stellar wind are one possible model to explain stationary X-ray emission in the jets of CTTS, but we cannot claim that recollimation shocks of stellar winds are the only possible explanation until all competing scenarios are analyzed to the same level of detail to be ruled out or confirmed. 




Acknowledgment: Support for this work was provided for HMG by NASA through grant GO-12907.01-A from the Space Telescope Science Institute, which is operated by the Association of Universities for Research in Astronomy, Inc., under NASA contract NASA 5-26555 and by NSF AST1313083 and NASA NNX14AB38G for ZYL.