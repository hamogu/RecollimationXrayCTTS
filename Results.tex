\section{Results}
\label{sect:results}
The last section already showed that for all parameters consistent with the theoretical and observational constraints the stellar wind is enclosed in a finite region by a shock front. This shock front generally reaches a maximum cylindrical radius of only several AUs, but a much larger height above the accretion disk for external pressure profiles with high pressure in the plane of the disk and a large pressure gradient (fiducial model in Fig.~\ref{fig:result}). A shallower pressure profile leads to a stellar wind region that is wider.So far, no stellar wind zone is resolved within the more massive and wider disk wind in any CTTS imaging, limiting the maximal size of the stellar wind region to a few AU. Our calculations show that this scenario is compatible with the known properties of the stellar wind. The biggest uncertainty is probably the value of the external pressure. As discussed above, different simulations in the literature predict similar pressure profiles, but the normalization of the pressure depends to a large degree on the disk magnetic field, which is only poorly constrained. In our calculation, we have scaled the pressure such that the post-shock densities are compatible with observations of the jet and we find a fiducial model that is compatible with the X-ray emission from the jet that we set out to explain. However, the magnitude of the pressure is a free parameter in our model and if it could be determined more accurately, that will confirm or rule-out the scenario we suggest in this article.

The highest post-shock temperatures are generally reached at the base of the jet when the stellar wind encounters the inner disk rim or at large $z$ when the shock front intersects the jet axis. Thus, the position of the hottest post-shock cooling plasma must be very close to the jet axis. In our fiducial model (Fig.~\ref{fig:result}, solid red line), the temperature is just sufficient to produce X-ray emission. Paper~I showed that in DG~Tau a small faction, about $10^{-3}$, of the total mass loss rate in the outflow is enough to power the observed X-ray emission at the base of the jet. Figure~\ref{fig:rhocool} shows the pre-shock number densities $n_0$ for the four models from Fig.~\ref{fig:result}. A detailed treatment of the post-shock region is beyond the scope of this paper, but an upper limit on the post-shock cooling length $d_{\mathrm{cool}}$ can be derived according to \citet{2002ApJ...576L.149R}:
\begin{equation}
d_{\mathrm{cool}} \approx 20.9 \mathrm{ AU}
    \left(\frac{10^5\mathrm{ cm}^{-3}}{n_0}\right)
    \left(\frac{v_{\mathrm{shock}}}{500\mathrm{ km s}^{-1}}\right)^{4.5}\ .
\end{equation}
The derivation for this formula assumes a cylindrical cooling flow. In contrast, in our model the external pressure will continue to compress the gas, as it starts cooling. Since denser gas emits more radiation and thus cools faster, $d_{\mathrm{cool}}$ is only an upper limit. With this in mind, figure~\ref{fig:rhocool} (lower panel) indicates that the cooling lengths for our fiducial model is consistent with the X-ray observations that do not resolve the wind shock \citep{2008A&A...488L..13S}. On the other hand, a model with a wind mass loss rate of only $10^{-10}$~M$_{\odot}$~yr$^{-1}$ violates the observational constraints. Since only a very small fraction of the stellar mass loss is heated to X-ray emitting temperatures (Fig.~\ref{fig:result}, rightmost panel) the low-mass loss scenario also does not provide enough X-ray luminosity to explain the observations (paper~I).
Significantly higher pressures require unrealistically fast outflows to push the shock front out to 40~AU and lower pressures do not allow a mass flux high enough to power the X-ray luminosity.

In our model, it is irrelevant how much of the external pressure is provided by the magnetic field in the disk wind and how much by thermodynamic pressure. The region of interest is still within the Alfv\'en surface (see references in Sect.~\ref{sect:boundary}), so the influence of the magnetic field probably dominates. Otherwise, the disk wind would also have to be very dense (probably too dense to be consistent with observations) to provide this pressure. Observationally, it is difficult to distinguish the stellar wind from the disk wind. The slower jet components observed further away from the jet axis carry much of the mass flow \citep{2000ApJ...537L..49B}. Their origin is probably the inner region of the disk and not the star \citep{2003ApJ...590L.107A}. Thus, it is fully consistent that our model predicts a mass loss fraction larger than  $10^{-3}$ of the stellar wind at X-ray emitting temperatures. If the disk wind dominates over the stellar wind in mass loss, then the fraction of hot gas in the (stellar plus inner disk) jet might still be small.