\section{Results}
\label{sect:results}
The last section already showed that for all parameters consistent with the theoretical and observational constraints the stellar wind is enclosed in a finite region by a shock front. This shock front generally reaches a maximum cylindrical radius of only several AUs, but a much larger height above the accretion disk. While a detailed numerical treatment of the post-shock cooling zone is beyond the scope of this work, the shape of the shock front indicates that the post-shock zone will also be rather narrow in $\omega$. The highest post-shock temperatures are generally reached at the base of the jet when the stellar wind encounters the inner disk rim or at large $z$ when the shock front intersects the jet axis. Thus, the position of the hottest post-shock cooling plasma must be very close to the jet axis. 

The temperature in our fiducial model stays just below 1~MK -- too little to explain X-ray emission in the jets (Fig.~\ref{fig:result}, solid red line), but small changes in the parameters, well within the observational and theoretical constraints, are sufficient to drive the maximal temperatures over 1~MK for a small fraction of the mass loss (other lines in the figure). 

Paper~I showed that a small faction, about $10^{-3}$, of the total mass loss rate in the outflow is enough to power the observed X-ray emission at the base of DG~Tau's jet. All but the fiducial scenario in Fig.~\ref{fig:result} have a significant, but small fraction of the stellar wind that gets heated to $>1.5$~MK and thus can easily emit X-rays. In this article, we concentrate on the stellar wind mass loss, but in the observations it is difficult to distinguish the stellar wind from the disk wind. The slower jet components observed further away from the jet axis carry much of the mass flow \citep{2000ApJ...537L..49B}. Their origin is probably the inner region of the disk and not the star \citep{2003ApJ...590L.107A}. Thus, it is fully consistent that our model predicts a mass loss fraction larger than  $10^{-3}$ of the stellar wind at X-ray emitting temperatures. If the disk wind dominates over the stellar wind in mass loss, then the fraction of hot gas in the (stellar plus inner disk) jet might still be small.