\section{Results}
\label{sect:results}

\subsection{Uncertainties}
All input parameters discussed above can take a range of values. We argue that the starting point of the integration has to be within the inner disk and the exact value has only a small influence on the result. The wind velocity is closely related to the temperature of the observed plasma, the mass loss rate to the size of the region and the luminosity.

The biggest uncertainty is probably the value of the external pressure. As discussed above, different simulations in the literature predict similar pressure profiles, but the normalization of the pressure depends to a large degree on the disk magnetic field, which is only poorly constrained. In our calculation, we have scaled the pressure such that the post-shock densities are compatible with observations of the jet and we find a fiducial model that is compatible with the X-ray emission from the jet that we set out to explain. However, the magnitude of the pressure is a free parameter in our model and if it could be determined more accurately, that will confirm or rule-out the scenario we suggest in this article.


\subsection{Size of stellar wind zone}
The last section already showed that for all parameters consistent with the theoretical and observational constraints the stellar wind is enclosed in a finite region by a shock front. This shock front generally reaches a maximum cylindrical radius of only several AUs, but a much larger height above the accretion disk for external pressure profiles with high pressure in the plane of the disk and a large pressure gradient (fiducial model in Fig.~\ref{fig:result}). A shallower pressure profile leads to a stellar wind region that is wider. 

Most of the imaging of YSO winds traces molecular lines and low-ionizations stages, e.g. \ion{O}{1} or \ion{Fe}{2}. These lines are formed in low-temperature regions, but not in a hot post-shock plasma. Thus, one could expect to see a hole that is filled by hot post-shock plasma form the stelalr wind. However, no such hole is resolved in any CTTS imaging. This limits the maximal size of the stellar wind region to a few AU. Our calculations show that this scenario is compatible with the known properties of the stellar wind. 

\subsection{X-ray luminosities}
The post-shock plasma is less dense than the typical stellar corona and can thus be treated in the so-called coronal approximation, meaning that the plasma is optically thin and line ratios for prominent X-ray lines are in the low-density limit. We use the shock models of \citet{http://adsabs.harvard.edu/abs/2007A%26A...466.1111G} to predict the fraction of the total pre-shock kinetic energy that will be emitted in the X-ray range. \citet{http://adsabs.harvard.edu/abs/2011AN....332..448G} published a grid of X-ray spectra\footnote{Available at http://hdl.handle.net/10904/10202} with pre-shock velocities between 300 and 1000~km~s$^{-1}$ in increments of 100~km~s$^{-1}$. We integrate all emission between 0.3 and 3~keV for each spectrum. At 300~km~s$^{-1}$ only 2\% of the available energy is emitted between 0.3 and 3~keV, so we set the fraction to zero for pre-shock velocities of 0, 100 and 200~km~s$^{-1}$, which are not covered by the model grid. The fraction of energy emitted in X-rays is independent of the density except for a few density-dependent emission lines with negligable contribution to the integrated flux. The physical size of the post-shock region depends strongly on the density, but


The highest post-shock temperatures are generally reached at the base of the jet when the stellar wind encounters the inner disk rim or at large $z$ when the shock front intersects the jet axis. Thus, the position of the hottest post-shock cooling plasma must be very close to the jet axis. In our fiducial model (Fig.~\ref{fig:result}, solid red line), the temperature is just sufficient to produce X-ray emission. Paper~I showed that in DG~Tau a small faction, about $10^{-3}$, of the total mass loss rate in the outflow is enough to power the observed X-ray emission at the base of the jet. 

\subsection{The size of the post-shock zone}
Figure~\ref{fig:rhocool} shows the pre-shock number densities $n_0$ for the four models from Fig.~\ref{fig:result}. A detailed treatment of the post-shock region is beyond the scope of this paper, but an upper limit on the post-shock cooling length $d_{\mathrm{cool}}$ can be derived according to \citet{2002ApJ...576L.149R}:
\begin{equation}
d_{\mathrm{cool}} \approx 20.9 \mathrm{ AU}
    \left(\frac{10^5\mathrm{ cm}^{-3}}{n_0}\right)
    \left(\frac{v_{\mathrm{shock}}}{500\mathrm{ km s}^{-1}}\right)^{4.5}\ .
\end{equation}
The derivation for this formula assumes a cylindrical cooling flow. In contrast, in our model the external pressure will continue to compress the gas, as it starts cooling. Since denser gas emits more radiation and thus cools faster, $d_{\mathrm{cool}}$ is only an upper limit. With this in mind, figure~\ref{fig:rhocool} (lower panel) indicates that the cooling lengths for our fiducial model is consistent with the X-ray observations that do not resolve the wind shock \citep{2008A&A...488L..13S}. Since only a very small fraction of the stellar mass loss is heated to X-ray emitting temperatures (Fig.~\ref{fig:result}, rightmost panel) the low-mass loss scenario also does not provide enough X-ray luminosity to explain the observations (paper~I).
Significantly higher pressures require unrealistically fast outflows to push the shock front out to 40~AU and lower pressures do not allow a mass flux high enough to power the X-ray luminosity.

