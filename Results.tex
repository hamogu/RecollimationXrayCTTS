\section{Results}
\label{sect:results}

\subsection{Uncertainties}
All input parameters discussed above can take a range of values. We argue that the starting point of the integration has to be within the inner disk and the exact value has only a small influence on the result. The wind velocity is closely related to the temperature of the observed plasma, the mass loss rate to the size of the region and the luminosity.

The biggest uncertainty is the value of the external pressure. As discussed above, different simulations in the literature predict similar pressure profiles, but the normalization of the pressure depends to a large degree on the disk magnetic field, which is only poorly constrained. In our calculation, we have scaled the pressure such that the post-shock densities are compatible with observations of the jet.


\subsection{Size of stellar wind zone}
For all parameters consistent with the theoretical and observational constraints the stellar wind is enclosed in a finite region by a shock front. This shock front generally reaches a maximum cylindrical radius of a few AUs, and a much larger height above the accretion disk for external pressure profiles with high pressure in the plane of the disk and a large pressure gradient (fiducial model in Fig.~\ref{fig:result}). A shallower pressure profile leads to a stellar wind region that is wider. 

\subsection{X-ray luminosities}
\label{sect:LX}
The post-shock plasma is less dense than the typical stellar corona and can thus be treated in the so-called coronal approximation, meaning that the plasma is optically thin and line ratios for prominent X-ray lines are in the low-density limit. We use the shock models of \citet{2007A&A...466.1111G} to predict the fraction of the total pre-shock kinetic energy that will be emitted in the X-ray range. We refer to that publication for details on the shock models. In summary, the models simulate radiative cooling of optically thin plasma in a two-fluid approximation, where electrons and ions are described with a Maxwellian velocity distribution, each with a different temperature. The ionization and recombination rates are calculated explicitly, but even for densities as low as $10^5$~cm$^{-3}$ only a small fraction of the post-shock cooling zone the inonization state differs significantly from the ionization equilibrium.

\citet{2011AN....332..448G} published a grid of X-ray spectra\footnote{Available at http://hdl.handle.net/10904/10202} based on these models with pre-shock velocities between 300 and 1000~km~s$^{-1}$ in increments of 100~km~s$^{-1}$. We integrate all emission between 0.3 and 3~keV for each spectrum. At 300~km~s$^{-1}$ only 2\% of the available energy is emitted between 0.3 and 3~keV (Figure~\ref{fig:fracxray}), so we set the fraction to zero for pre-shock velocities of 0, 100 and 200~km~s$^{-1}$, which are not covered by the model grid. The fraction of energy emitted in X-rays is independent of the density except for a few density-sensitive emission lines with negligible contribution to the integrated flux. The physical size of the post-shock region depends strongly on the density, but total energy available only depends on the pre-shock velocity and the total mass flux. Thus, the X-ray luminosity $L_X$ does not change, if the post-shock region is compressed by some external pressure.

The highest post-shock temperatures are generally reached at the base of the jet when the stellar wind encounters the inner disk rim or at large $z$ when the shock front intersects the jet axis. In our fiducial model (Fig.~\ref{fig:result}, solid red line), the pre-shock velocity is $>250$~km~s$^{-1}$ at $z<5$~AU and $z>20$~AU. Given the large solid angle covered by the inner disk rim, the  $z<5$~AU region contributes significantly to the total mass flux (compare the red line and the red filled histogram in Figure~\ref{fig:result}, rightmost panel). However, in most YSOs the central object is highly absorbed. Therefore, we calculate all $L_X$ values taking into account only regions with  $z>5$~AU. For the fiducial, the high $v_\infty$, the low $\dot M$, and the shallow $P$ model in Figure~\ref{fig:result} the predicted $L_X$ is $3\cdot10^{29}$, $5\cdot10^{30}$, $1\cdot10^{28}$, and $1\cdot10^{31}$~erg~s$^{-1}$, respectively.
\citet{2009A&A...493..579G} already showed that in DG~Tau a small fraction, about $10^{-3}$, of the total mass loss rate in the outflow is enough to power the observed X-ray emission at the base of the jet. In our fiducial model, this small fraction corresponds to the mass flow close to the jet axis, where the pre-shock velocities are highest.

\subsection{The size of the post-shock zone}
Figure~\ref{fig:rhocool} shows the pre-shock number densities $n_0$ for the four models from Fig.~\ref{fig:result}. A detailed treatment of the post-shock region is beyond the scope of this paper, but an estimate for the post-shock cooling length $d_{\mathrm{cool}}$ can be derived according to \citet{2002ApJ...576L.149R}:
\begin{equation} \label{eqn:dcool}
d_{\mathrm{cool}} \approx 20.9 \mathrm{ AU}
    \left(\frac{10^5\mathrm{ cm}^{-3}}{n_0}\right)
    \left(\frac{v_{\mathrm{shock}}}{500\mathrm{ km s}^{-1}}\right)^{4.5}\ .
\end{equation}
The derivation for this formula assumes a cylindrical cooling flow. In contrast, in our model the external pressure can change with $z$ and might compress the post-shock gas or allow it to expand radially. Since denser gas emits more radiation and thus cools faster, $d_{\mathrm{cool}}$ is only an estimate. However, for our models $P(z)$ has already reached $P_\infty$ at the position of the shock and thus eqn.~\ref{eqn:dcool} is valid. With this in mind, figure~\ref{fig:rhocool} (lower panel) indicates that the cooling lengths for our fiducial model is consistent with the X-ray observations that do not resolve the wind shock \citep{2008A&A...488L..13S}.
