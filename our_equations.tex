\section{The model}
\label{sect:model}
In this section we develop an analytical steady-state model for the interface between the stellar wind and the surrounding disk wind. The pressure of the disk wind collimates the stellar wind into a jet (see figure~\ref{fig:sketch} for a sketch). The two flows are separated by a contact discontinuity, whose exact position is given by pressure equilibrium between the outer, disk wind component and the inner, stellar wind component. As the stellar wind encounters the contact discontinuity, the velocity component perpendicular to the discontinuity is shocked. Thus, our model needs to distinguish three zones: (i) the cold pre-shock stellar wind, (ii) the hot post-shock stellar wind, and (iii) the disk wind. Our goal is to calculate the geometrical shape of the stellar wind shock, since this determines the velocity jump across the shock front and the temperature of the post-shock plasma. 

\subsection{The shape of the shock front}

The Rankine-Hugoniot jump conditions relate the mass density $\rho$, velocity $v$, and pressure $P$ on both sides of a shock. For ideal gases and non-oblique shocks the conservation of mass, momentum and energy across the shock can be written as follows \citep[][chap.~7]{1967pswh.book.....Z}, where the state before the shock front is marked by the index 0, that behind the shock by index 1:
\begin{eqnarray}
\rho_0 v_0 & = & \rho_1 v_1 \label{eqn:RH1}\\
\label{eqn:RH2}P_0+\rho_0 v_0^2 & = & P_1+\rho_1 v_1^2\\
\label{eqn:RH3}\frac{5 P_0}{2\rho_0}+\frac{v_0^2}{2}& = &\frac{5 P_1}{2\rho_1}+\frac{v_1^2}{2} \ .
\end{eqnarray}

We assume that the stellar wind before the shock front is relatively cool \textbf{(this assumption is justified in Section~\ref{sect:T_0})} and thus the thermodynamic pressure can be neglected, setting $P_0=0$.
The shock front settles at a position where the pressure of the stellar wind equals the post-shock pressure, which in turn determines the position of the contact discontinuity, such that the post-shock pressure equals the confining external pressure of the disk wind $P(z)$ . 

In our case we are dealing with an oblique shock (Figure~\ref{fig:sketch}). Equations~\ref{eqn:RH1} to \ref{eqn:RH3} stay valid if only the velocity component perpendicular to the shock front is taken as $v_0$. 

We use a cylindrical coordinate system $(z, \omega, \theta)$ with an origin at the central star. We place the $z$-axis along the jet outflow direction and assume rotational symmetry around the jet axis. Thus, the flow can effectively be written in $(z,\omega)$. The symbol $r$ denotes the spherical radius, i.e.\ the distance of any point to the star at the origin of the coordinate system. 
\textbf{In this article we adapt the model of 
\citet{2012MNRAS.422.2282K} to non-relativistic speeds. Figure~1 in their publication shows the geometry of this model in much detail} and we refer to their discussion and their figure~1 for a more detailed description. \textbf{Although the basic model is the same, we chose to include a similar figure (our Fig.~\ref{fig:sketch}) in this work for the benefit of readers who are not familiar with the work of \citet{2012MNRAS.422.2282K} on extra-galactic jets.} 

We treat the disk wind as an outer boundary condition with a given pressure profile and concentrate on the description of the stellar wind. To simplify the equations we adopt Kompaneets' approximation \citep{1960SPhD....5...46K} which states that there is no axial pressure gradient so that the pressure profile of the disk wind extends through all layers of the outflow:
\begin{equation}
P(z, \omega, \theta) = P(z)\,.
\end{equation}
With this we can write:
\begin{equation}\label{eqn:Pofz}
\rho_0 v_0^2 = P_{\textrm{post-shock}} = P(z)
\end{equation}
