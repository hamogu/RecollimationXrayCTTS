\subsection{Launching radius}
From a mathematical point of view, the starting point of the integration in the plane of the disk can be chosen freely anywhere between $\omega=0$ and $\omega=R_0(z=0)$. Figure~\ref{fig:omega_0} compares differnet starting point under otherwise equal conditions. For small inital radii the ram presure of the stellar wind pushes the shock surface out to larger radii in a comparatively small $\Delta z$. This leads to small pre-shock speeds in this region because the direction of the flow and the shock surface are almost parallel. This region also represents a large fraction of the total mass loss of the stellar wind, because it covers a large angle in the $(z,\omega)$-plane and a large solid angle of the spherical wind emission. Consequently, models with small values for $\omega_0$ show much less material that is heated up high temperatures. 

Physically, the position of the shock front is restricted by the position of the disk - the shock between the stellar wind and the disk material (in the disk itself or the disk wind) must occour within the inner hole of the disk. Fortunately, figure~\ref{fig:omega_0} shows that the two solutions for $\omega_0=0.01$~AU and 0.1~AU are almost indistinguishable and the extact value for this parameter is not important as long as it is small. From here on we use $\omaga_0 = 0.01$~AU.