\subsubsection{Magnetic fields}
Two different regions need to be distinguished where magnetic fields can play a role. First, a magnetic field can provide additional pressure in the disk wind and thus contribute to the external pressure $P(z)$. Indeed, the region covered by our model is expected to be inside the Alv\`en surface of the disk wind, so $P(z)$ is probably magnetically dominated. This will be discussed in more detail in Section~\ref{sect:boundary}, but for our model only the total value of $P(z)$ matters, independent of the processes that contribute to the pressure. 

Second, the stellar wind could be threaded by a stellar magnetic field. Fields on YSOs are often quite complex with a mixture of open and closed field lines \citep[e.g.][]{2011MNRAS.417..472D,2012MNRAS.425.2948D} and their configuration changes in coronal activity.
Qualitatively, closed field lines can either fill with coronal plasma or connect to the accretion disk and carry accretion funnels. Only those parts of the stellar surface connected to open field lines can launch a wind. Thus, the total mass loss rate would be reduced compared to a spherical wind that we assume here. 
As a simple estimate we calculate the magnetic pressure $P_{\textrm{mag}}=\frac{\vec B^2}{8 \pi}$ for a split monopole field with a field strength of 1~kG at $r=R_\odot$ and compare it to the ram presure (eqn.~\ref{eqn:Pofz}). Using the fiducial parameters from Table~\ref{tab:fiducial} the ram presure dominates over the magnetic pressure already at 0.1~AU and since $P_{\textrm{mag}} \propto \vec B^2 \propto r^{-4}$, while $P_\{textrm{ram}} \propto r^{-2}$ (eqn.~\ref{eqn:Pofz} and \ref{eqn:rho}) we can neglect the magnetic presure of the stellar wind for our model.



%Otherwise, the disk wind would also have to be very dense (probably too dense to be consistent with observations) to provide this pressure. Observationally, it is difficult to distinguish the stellar wind from the disk wind. The slower jet components observed further away from the jet axis carry much of the mass flow \citep{2000ApJ...537L..49B}. Their origin is probably the inner region of the disk and not the star \citep{2003ApJ...590L.107A}. Thus, it is fully consistent that our model predicts a mass loss fraction larger than  $10^{-3}$ of the stellar wind at X-ray emitting temperatures. If the disk wind dominates over the stellar wind in mass loss, then the fraction of hot gas in the (stellar plus inner disk) jet might still be small.