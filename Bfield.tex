\subsubsection{Magnetic fields}
Two different regions need to be distiguished where magnetic fields can play a role. First, a magnetic field can provide additional presure in the disk wind and thus contribute to the external presure $P(z)$. Indeed, the region covered by our model is expected to be inside the Alv\`en surface of the disk wind, so $P(z)$ is probably magntically dominated. This will be discussed in more detail in Section~\ref{sect:boundary}, but for our model only the total value of $P(z)$ matters, independent of the processes that contribute to the pressure. 

Second, the stellar wind could be threaded by a stellar magnetic field. Fields on YSOs are often quite complex with a mixture of open and closed field lines \citep[e.g.][]{2011MNRAS.417..472D,2012MNRAS.425.2948D} and their configuration changes in coronal activity. Since field strenght and geometry on YSOs are variable and not well constrained, we cannot include this effect in our model.

Qualitatively, closed field lines can either fill with coronal plasma or connect to the accretion disk and carry accretion funnels. Only those parts of the stellar surface connected to open field lines can launch a wind. Thus, the total mass loss rate would be reduced compared to a spherical wind that we assume here. On the other hand, the field would provide some additional pressure, compensating for the loss of thermal presure due to a reduced density.