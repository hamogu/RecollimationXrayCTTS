\section{Introduction} 
In many areas of astrophysics compact central objects accrete mass and angular momentum from a disk and at the same time they eject a highly collimated jet. This is seen for central objects as massive as AGN or as light as (proto) brown dwarfs. Jets from AGN or accreting neutron stars reach relativistic velocities while the material in jets from young stellar objects (YSOs) is significantly slower. 
A large number of jets is observed in nearby star forming regions, where the jet composition and structure can be studied in great detail \citep[see the review by][]{2014arXiv1402.3553F}.
Jet launching starts in the early stages of star formation and continues until the accretion from the circumstellar disk cedes. Jets from very young stars are the most powerful and can sometimes be traced up to several parsec from the source. In these systems, the central engine is still deeply embedded in a dense envelope of gas and dust and thus cannot be observed directly. As YSOs evolve, the envelope becomes thinner. Actively accreting low-mass stars in this stage are called classical T Tauri stars \citep[for a review see][]{2013AN....334...67G}. Their jets often only reach a few hundred AU (and are thus sometimes called ``microjets'' in comparison to the outflows from younger objects).
It seems reasonable to suspect that the same physics governs the launching from any type of young star and that the same processes occur close to the launch site, but observationally the inner few tens of AU are only accessible in CTTS, where we can study the initial properties before the jet interacts with the ambient medium. 

Recently, there has been increasing evidence that jets from CTTS have a stationary hot (in the MK range), X-ray emitting region only a few tens of AU from the central star (Section~\ref{sect:introxray}). In this article we want to establish recollimation boundary layers between the stellar wind and a disk wind as a viable scenario to explain stationary X-ray and UV emission from YSO jets.
While X-ray emission has been discussed in the literature (Section~\ref{sect:intromodel}), stationary shocks between a stellar wind and a recollimating disk wind have not been investigated in detail. 

X-rays trace the fastest and most energetic components of jets. They can also influence the chemistry deep in the disk \citep[e.g.][]{2010ApJ...714.1511H,2012ApJ...756..157G} and thus alter the environment of planet formation because they penetrate deeper than UV and optical radiation. Unlike stellar X-ray emission, the radiation from the jet originates above the plane of the disk and thus reaches the entire disk surface, while stellar radiation may be shadowed by the inner disk rim.

In the remainder of the introduction we review observational properties (including X-ray emission) of jets from CTTS and summarize theoretical explanations for this emission in the literature. In section~\ref{sect:model} we develop the equations that govern a standing shock front and discuss the physical parameters in section~\ref{sect:parameters}. In section~\ref{sect:results} we present our results and discuss implications in section~\ref{sect:discussion}. We summarize this work in section~\ref{sect:summary}.