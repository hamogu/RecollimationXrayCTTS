\section{Introduction} 
In many areas of astrophysics compact central objects accrete mass and angular momentum from a disk and at the same time they eject a highly collimated jet. This is seen for central objects as massive as AGN or as light as (proto) brown dwarfs. For objects like AGN or accreting neutron stars the jets reach relativistic energies while the velocities are significantly lower in young stellar systems. 
A large number of jets is observed in near star forming regions, where the jet composition and structure can be studied in great detail \citep[see the review by][]{2014arXiv1402.3553F}.
Jets are launched from the early stages of star formation until the accretion from the circumstellar disk ceeds. Jets from very young stars (class I) are the most powerful and can be traced for long distances up to a parsec from the source, but the central engine is still deeply embedded in a dense envelope of gas and dust and thus cannot be observed directly. As the young stellar objects evolve, mass from the envelope becomes thinner and accretes onto the circumstellar disk. In this stage young, low-mass stars that actively accrete from their circum-stellar disk are called classical T Tauri stars \citep[for a review see][]{2013AN....334...67G}. Their jets often only reach a few hundred AU (and are thus sometimes called ``mircojets'' in comparison to the outflows from younger objects), but the lower column density makes the jet acessibly to observations just a few tens of AU from the central star.

It seems reasonable to suspect that the same physics governs the launching from any type of young star and that the same processes occur close to the lauch site, but observationally the inner few tens of AU are only accessable in CTTS, thus we need to concentrate on CTTS jets to study the initial properties before a jet interacts with the ambient medium. Recently, there has been increasing evidence that jets from CTTS have a stationary hot (in the MK range) emission region only a few tens of AU from the central star (Section~\ref{sect:introxray}). In this article, we want to study the scenario of a starionary recollimation shock as an explanation for this component. While X-ray emission has been discussed in the literature (Section~\ref{sect:intromodel}), starionary recollimation shocks have not been investigated in detail as sources of high-energy radiation. X-rays trace the fastest and most energetic components of jets. They can also influence the chemistry deep in the disk \citep[e.g.][]{2010ApJ...714.1511H,2012ApJ...756..157G} because they penetrate deeper than UV and optical radiation and thus alter the environment of planet formation. Unlike stellar X-ray emission, the radiation from the jet originates above the plane of the disk and thus reaches the entire disk surface, while stellar radiation may be shadowed by the inner disk rim.

In the remainder of the introduction we review observational properties of jets from CTTS and summarize theoretical explantions for this emission in the literature. In section~\ref{sect:model} we develop the equations that govern the standing shock front and discuss the physical parameters in section~\ref{sect:parameters}. In section~\ref{sect:results} we present our results and discuss implications in section~\ref{sect:discussion}. We summarize this work in section~\ref{sect:summary}.