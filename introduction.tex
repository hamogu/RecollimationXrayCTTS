\section{Introduction} 
In many areas of astrophysics massive central objects accrete mass and angular momentum from a disk and at the same time they eject a highly collimated jet. This is seen for central objects as massive as AGN or as light as (proto) brown dwarfs. For the most massive and most compact objects like AGN or accreting neutron stars the jets reach relativistic energies while the velocities are significantly lower in young stellar systems. 

Young, low-mass stars that actively accrete from a circum-stellar disk are called classical T Tauri stars. The slowest velocities are observed in molecular lines with typical line shifts of only a few km~s$^{-1}$ \citep{2008ApJ...676..472B}. These molecular outflows have wide opening angles around 90\degree{} \citep[e.g.][]{2013A&A...557A.110S} and are presumably launched from the disk. Faster components are seen in optical emission lines like H$\alpha$ or in forbidden emission lines such as [\ion{O}{1}] or [\ion{S}{2}]. \citet{2000ApJ...537L..49B} observed the jet from the CTTS \object{DG Tau} with seven long-slit exposures of \emph{HST}/STIS to resolve the kinematic structure of the jet both along and perpendicular to the jet axis. They find that the faster jet components are better collimated and propose an ``onion'' scenario, where the fastest jet components make up the innermost layer and the surrounding layers have lower velocities the further away from the jet axis they are. The fastest components seen in the optical emission lines are typically 200-300~km~s$^{-1}$ \citep{2004Ap&SS.292..651B,2008ApJ...689.1112C,2013A&A...550L...1S}.

Yet, in some jets from CTTS there is evidence for another component which is even more energetic. The best studied case is DG~Tau that was the target of several shorter \emph{Chandra} exposures in 2004, 2005, and 2006 and a large program in 2010 \citep{2005ApJ...626L..53G,2008A&A...478..797G,2011ASPC..448..617G}. These observations showed X-ray emission from three distinct regions: First, weak and soft emission from the jet is resolved several hundred AU from the star itself. Second, hard emission from the central star is observed with stellar flares as seen on many other young and active stars. However, since the star itself is embedded in circumstellar material, the soft X-ray are absorbed. Instead, soft X-rays come from a region about 30-40~AU above the plane of the accretion disk. Their peaks are consistent with a position on the jet axis, but the uncertainties on the position would also allow an off-axis emission region \citep{2008A&A...488L..13S}. The luminosity and temperature of this inner emission region are remarkably stable over one decade. The maximum change observed is XXX \citep{SchneiderDGTauXray}.

In \citet{2009A&A...493..579G} we showed that this inner X-ray emission can be explained by shock heating of a 400-500~km~s$^{-1}$ fast component. For the case of DG~Tau the mass flux in this component can be less than $10^{-4}$ of the total mass flux in the jet or even lower if the same material is reheated in several consecutive shocks. If the density in the fast outflow is $>10^5$~cm$^{-3}$ then the cooling length of this shock is only a few AU and thus it would be unresolved and outshined by the more luminous emission from the more massive, but slightly slower jet component, that is responsible for the optical emission.

In this article we explain how such a shock can be caused by the recollimation of the inner jet due to the shape of the boundary between stellar winds and disk winds similar to the work of \citet{2012MNRAS.422.2282K} for relativistic jets. In section~\ref{sect:model} XXX. We end with a summary in section~\ref{sect:summary}.