\subsection{Models for high-energy emission from stellar jets}
\label{sect:intromodel}
When \citet{2001Natur.413..708P} discovered X-ray emission in HH~2, they immidiately discussed strong shocks in the outflow as the most likely heating process and the most obvious way to generate a strong shock is a bow shock at the head of the outflow, where the jet encounters circumstellar material. Depneding on the density contrast between the jet and the ambient material, the heating can occur in either the forward shock that is driven into the ISM or in the reverse shock that travels against the direction of the flow in the jet. However, in most YSO jets, we see the X-ray emission fairly close to the source, where the ISM has been cleared by outflow activity a long time ago. \citet{2003ApJ...584..843B} study the X-ray emission from \object{L1551 IRS 5}, a binary protostar. They suggest several classes of models. First, the observed X-rays might be stellar emission, that is scattered into our line-of-sight by circumstellar material. This requires high densities to reach the required scattering efficiency and the material must be neutral, otherwise it would show up as bright radio emission, which is not seen to be coincident with the X-ray detection in YSO jets. Second, a time variable launching velocity would cause shocks to propagate along the jet, continuously heating the jet material they encounter. This scenario is confimed for a shock in the jet of HD~163296 \citep{2013A&A...552A.142G}, but the velocity differences are too small to heat the jet to X-ray emitting temperatures. Third, fast shocks form when the jet encounters and almost stationary obstacle. In the absence of dense ISM and the positon of the X-ray emitting regions this could be a collimation shock, the disk and the outflow of another star. Jets are typically ejected roughly perpendicular to the disk \citep[e.g.][for IRS 5]{http://adsabs.harvard.edu/abs/2002A&A...382..573F}, making an interaction with the disk surface unlikely. The interaction with the outflow of a second YSO on scales of a few hundred AU only works in binary system, such as IRS~5, but X-rays are also dected from stars that are apparently single, such as DG~Tau and RY~Tau. A forth possibility, again only for binary systems, is magnetic heating from the reconnection between two interacting outflows as suggested by \citet{http://adsabs.harvard.edu/abs/2008A%26A...478..453M}.

In this article, we will analytically explore how a recollimation shock can explain stationary X-ray emission from jets. Based on the observational evidence for multi-layered jets, we assume that the innermost component of the flow is a stellar wind, which is collimated by the disk wind, which feeds the outer layers of the flow. Collimation shocks of this kind have not been treated in detail in the literature, while X-ray emission due to a moving shocks has been studied by several authors \citep[see, e.g.\ the analytical work and numerical simulations by][]{http://adsabs.harvard.edu/abs/2002ApJ...576L.149R,http://adsabs.harvard.edu/abs/2007A%26A...462..645B,http://adsabs.harvard.edu/abs/2010A%26A...517A..68B}.

\citet{http://adsabs.harvard.edu/abs/2011ApJ...737...54B} numerically simulated stationary X-ray shocks. To do so, they impose a rigid nozzle with a radius of a few hundred AU and inject a flow of plasma with a flat velocity profile along the jet axis. They find that a diamond shock forms at a hight roughly twice the radius of the nozzle, where the temprature are high enough to explain the X-ray emission ffrom HH~154. In contrast to that work, we do not impose rigid boundaries that collimate the flow, but instead prescribe an external presure profile and then calculate the position of the boundary between the inner wind and the external medium. Also, the setup of \citet{http://adsabs.harvard.edu/abs/2011ApJ...737...54B} is well-suited to study regions at larger distance from the star, but in this article we concentate on the inner few AU, where the outflow is not yet parallel to the jet axis. Instead, we start with a spherical flow from the stellar surface.

In this article we explain how such a shock can be caused by the recollimation of the inner jet due to the shape of the boundary between stellar winds and disk winds similar to the work of \citet{2012MNRAS.422.2282K} for relativistic jets. This scenario naturally explains the stationary appearance and its location within the jet collimation region.