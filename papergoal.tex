\subsection{Models for high-energy emission from stellar jets}
\label{sect:intromodel}
When \citet{2001Natur.413..708P} discovered X-ray emission in HH~2, they immediately discussed strong shocks in the outflow as the most likely heating process and the most obvious way to generate a strong shock is a bow shock at the head of the outflow, where the jet encounters circumstellar material. Depending on the density contrast between the jet and the ambient material, either the forward shock that is driven into the ISM or the reverse shock that travels against the flow in the jet possesses higher shock velocities. However, in most YSO jets, we see the X-ray emission fairly close to the source, where the ISM has been cleared by outflow activity a long time ago. \citet{2003ApJ...584..843B} study the X-ray emission from \object{L1551 IRS 5}, a binary protostar. They suggest several classes of models. First, the observed X-rays might be stellar emission, that is scattered into our line-of-sight by circumstellar material. This requires high densities to reach the required scattering efficiency and the material must be neutral, otherwise it would show up as bright radio emission, which is not seen coincident with the X-ray detection in L1551~IRS~5 \citep{2003ApJ...584..843B}. In DG~Tau this scenario can be excluded because the lightcurve for the soft X-ray emission is flat, while coronal flares are seen in the hard-X-rays \citep{2011ASPC..448..617G}. If the soft X-rays were scattered stellar emission, they should display the same lightcurve.

Second, a time variable launching velocity would cause shocks to propagate along the jet, continuously heating the jet material they encounter \textbf{\citep[e.g.][]{2010A&A...511A..42B,2010A&A...517A..68B}}. This scenario is confirmed for an optical knot in the jet of HD~163296 \citep{2013A&A...552A.142G}, but the velocity differences are too small to heat the jet to X-ray emitting temperatures. Third, fast shocks form when the jet encounters an obstacle. In the absence of dense ISM this could be a collimation shock, the disk, or the outflow of another star. Jets are typically ejected roughly perpendicular to the disk \citep[e.g.][for IRS 5]{2002A&A...382..573F}, making an interaction with the disk surface unlikely. The interaction with the outflow of a second YSO on scales of a few hundred AU only works in binary systems, such as IRS~5, but X-rays are also detected from stars that are apparently single, such as DG~Tau and RY~Tau. A forth possibility is magnetic heating from the reconnection either between two interacting outflows \citep{2008A&A...478..453M} or from fields that are frozen into the outflow \citep{2013A&A...550L...1S}.

In this article, we will analytically explore how a recollimation shock can explain stationary X-ray emission from jets. Based on the observational evidence for multi-layered jets, we assume that the innermost component of the flow is a stellar wind, which is collimated by the disk wind. Collimation shocks of this kind have not been treated in detail in the literature, while X-ray emission due to a moving shock has been studied by several authors \citep[see, e.g.\ the analytical work and numerical simulations by][]{2002ApJ...576L.149R,2007A&A...462..645B}.

\textbf{\citet{2010A&A...511A..42B} presented simulations with a time variable launching speed, where blobs of material are emitted into the jet every few months or years. Their jet has a radial velocity profile that avoids the growth of random perturbations at the jet boundary and that is compatible with the expected magnetic fields in the environment of young stars. In these simulations faster material catches up with slower, previously emitted matter and shocks form that travel along the jet. Since material is launched almost continuously, the first interaction often takes place fairly close to the star and the simulations show a X-ray emission region only about 100~AU from the star, which fluctuates in luminosity but is present at all times. This region represents only a small fraction of the total simulated X-ray emission from the jet \citep{2010A&A...517A..68B}. }

\citet{2011ApJ...737...54B} numerically simulated stationary X-ray shocks. To do so, they impose a rigid nozzle with a radius \textbf{between 15 and 200 AU} and inject a flow of plasma with an \textbf{intially} flat velocity \textbf{and density} profile along the jet axis. The region that accelerates the mass at the bottom of the nozzle is not part of the model, but given the large radius, both disk wind and stellar wind might contribute in such a scenario. \citet{2011ApJ...737...54B} find that \textbf{a denser layer forms on the walls of the nozzle and that this perturbation travels inward. When this feature reaches the axis of symmetry} a diamond-shaped shock forms at a height \textbf{of 200-300 AU for a nozzle with a radius of 100~AU} with temperatures high enough to explain the X-ray emission from HH~154. \textbf{This model has not been applied to DG Tau, but might provide a viable explanation for the emission in DG Tau, too, if the shape and size of the nozzle is tuned properly.} 

In contrast to that work, we do not impose rigid boundaries that collimate the flow, but instead prescribe an external pressure profile and then calculate the position of the boundary between the inner wind and the external medium. The setup of \citet{2011ApJ...737...54B} is well-suited to study regions at \textbf{a distance of 200-300~AU} from the star, but in this article we concentrate on the inner region, where the outflow is not yet parallel to the jet axis and stellar and disk outflows have different velocities. Thus, we start with a spherical flow from the stellar surface and explain how a shock can be caused by the recollimation of the stellar outflow due to pressure from the outer disk winds. 
\textbf{\citet{2012MNRAS.422.2282K} developed a model for this geometry in the context of relativistic jets. Here we apply this model to stellar jets.}