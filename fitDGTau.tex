\subsection{Fit to DG Tau data}
In this section we perform a formal fit of our model to the \emph{Chandra} data of DG~Tau. There are some limitations inherent in the model (see discussion in Section~\ref{sect:modelassumptions}) and in the fitting process (see below), but the resulting best-fit model reproduces the observations with parameters that are in line with the observational limits discussed above and thus shows that our model is consistent with the data.

We download and extract the following\emph{Chandra} ObdIDs form the archive: \dataset[ADS/Sa.CXO#obs/04487]{4487}, \dataset[ADS/Sa.CXO#obs/06409]{6409}, \dataset[ADS/Sa.CXO#obs/07246]{7246}, and \dataset[ADS/Sa.CXO#obs/07247]{7247}. These datasets have been used and are described in \citet{2008A&A...478..797G,2008A&A...488L..13S} and paper~I and we refer the reader to those publications for more details. We processed the data with CIAO~4.6 \citep{http://adsabs.harvard.edu/abs/2006SPIE.6270E..60F}, extracting the CCD spectrum from DG~Tau and a larger, source-free background region on the same chip with the ``specextract'' script. Because the number of counts in the soft component is low and we are not interested in the rapid changes seen in the hard emission attributed to the stellar corona \citep{2008A&A...478..797G} we combine all four source spectra. We bin them to 25 counts per bin and subtract the background. We fit a model with two thermal optically thin plasma emission components \citep[APEC][]{http://adsabs.harvard.edu/abs/2012ApJ...756..128F} each with its own cold absorber analogous to \citep{2008A&A...478..797G} in the Sherpa fitting tool. We then replace the cooler APEC component by our recollimation shock model. Since the numerical evaluation of this model is slow, we fix the properties of the hot component at the values obtained in the privious fit to reduce the number of free parameters. To further reduce the number of parameters, we set $P_0 = 100\times P_\infty$ and $h=5$~AU. For each set of parameters, our model is evaluated as follows: We solve the ODE in eqn.~\ref{eqn:ode} numerically as above and calculate mass flux and pre-shock velocity for each numerical step. To take the high absorbing column density to DG~Tau itself into account we discard all step with $z<5$~AU. We bin the remaining mass flux according to the pre-shock velocity in each step in bins of 250-350, 350-450, ... 950-1050~km~s$^{-1}$. For each bin we select the appropriate post-shock cooling spectrum from the model grid discussed in Section~\ref{sect:LX} and scale it with the mass flux and an assumed distance to DG~Tau of 140~pc \citep{http://adsabs.harvard.edu/abs/1994AJ....108.1872K}. We use Sherpa to adjust $\dot M$, $P_0$, $v_\infty$, and the absorbing column density $N_\textrm{H}$ to simultaneously reproduce the observed X-ray spectrum and the position of the shock. \citet{2008A&A...488L..13S} and \citet{2011ASPC..448..617G} give distances of 25-45~AU between DG~Tau and the soft X-ray emission, but do not calculate formal errors for the position. For the purpose of a $\chi^2$ fit we compare the position where the shock front comes back to the jet axis to the value $z_{max} = 30\pm5$~AU.

The best-fit parameters for the shock model are given in table~\ref{tab:fiducial}, $N_\textrm{H}=X$ for the shock, and the parameters of the hot coronal component are $N_\textrm{H}=X$, plasma temperature $kT = X $~keV, and volume emission measure $VEM=X$. The best-fit has $z_{max} = X$ and $\chi^2_{red}= X$. The fitted spectrum is shown in Figure~\ref{fig:fit}.