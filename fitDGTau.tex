\subsection{Fit to DG Tau data}
In this section we perform a formal fit of our model to the \emph{Chandra} data of DG~Tau. 
Given the limitations of the model, there are some systematic uncertainties in the fitted values. Nevertheless the model reproduces the observations with parameters that are in line with the observational limits discussed above and thus shows that our model is consistent with the data.

We download and extract the following \emph{Chandra} observations from the archive: ObsIDs \dataset[ADS/Sa.CXO#obs/04487]{4487}, \dataset[ADS/Sa.CXO#obs/06409]{6409}, \dataset[ADS/Sa.CXO#obs/07246]{7246}, and \dataset[ADS/Sa.CXO#obs/07247]{7247}. These datasets have been used and are described in \citet{2008A&A...478..797G}, \citet{2008A&A...488L..13S}, and \citet{2009A&A...493..579G} and we refer the reader to those publications for more details. We processed the data with CIAO~4.6 \citep{2006SPIE.6270E..60F}, extracting the CCD spectrum from DG~Tau and a larger, source-free background region on the same chip with the \texttt{specextract} script.  \citet{2008A&A...478..797G} showed that the spectral properties of the soft X-ray component are compatible within the errors in all four observations and \citet{2008A&A...488L..13S} demonstrate that the offsets measured between the soft and the hard component is compatible as well. 
Because the number of counts in the soft component is low and we are not interested in the rapid changes seen in the hard emission attributed to the stellar corona \citep{2008A&A...478..797G} we combine all four source spectra. We bin them to 25 counts per bin and subtract the background. We fit a model with two thermal optically thin plasma emission components \citep[APEC,][]{2012ApJ...756..128F} each with its own cold absorber analogous to \citet{2008A&A...478..797G}. We then replace the cooler APEC component by our recollimation shock model. Since the numerical evaluation of this model is slow, we fix the properties of the hot, stellar component at the values obtained in the previous fit to reduce the number of free parameters. To further reduce the number of parameters, we set $P_0 = 100\times P_\infty$ and $h=5$~AU. For each set of parameters, our model is evaluated as follows: We solve the ODE in eqn.~\ref{eqn:ode} numerically and calculate mass flux and pre-shock velocity for each numerical step. To take the high absorbing column density to DG~Tau itself into account we discard all steps with $z<5$~AU. We bin the remaining mass flux according to the pre-shock velocity in bins of 250-350, 350-450, ..., 950-1050~km~s$^{-1}$. For each bin we select the appropriate post-shock cooling spectrum from the model grid discussed in Section~\ref{sect:LX} and scale it with the mass flux and an assumed distance to DG~Tau of 140~pc \citep{1994AJ....108.1872K}. We use the Sherpa fitting tool \citep{2001SPIE.4477...76F} to adjust $\dot M$, $P_0$, $v_\infty$, and the absorbing column density $N_\textrm{H}$ along the line-of-sight to the shock to simultaneously reproduce the observed X-ray spectrum and the position of the shock. \citet{2008A&A...488L..13S} and \citet{2011ASPC..448..617G} give distances of 25-45~AU between DG~Tau and the soft X-ray emission, but do not calculate formal errors for the position. For the purpose of a $\chi^2$ fit we compare the position where the shock front intersects the jet axis to the value $z_{max} = 30\pm5$~AU.

The best-fit parameters for the shock model are given in table~\ref{tab:fiducial}, $N_\textrm{H}=(4.7\pm0.3)\cdot10^{21}\mathrm{ cm}^{-2}$ for the shock, and the parameters of the hot coronal component are $N_\textrm{H}=2.6\times10^{22}$~cm$^{-2}$, plasma temperature $kT = 2.2$~keV, and volume emission measure $VEM=5\times10^{52}$~cm$^{-5}$. The best-fit has $z_{max} = 30.4$~AU and $\chi^2_{red}= 1.1$. The fitted spectrum is shown in Figure~\ref{fig:fit} where the black dots with error bars represent the data, the red line shows the full model, and the orange lines show the individual model components. The shock and coronal component dominates at low and high energies respectively.