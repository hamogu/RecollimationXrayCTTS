To derive the position of the shock front in the $(z, \omega)$ plane where the pre-shock ram pressure of the stellar wind and the post-shock pressure equal the external pressure $P(z)$, we need to calculate the pre-shock density $\rho_0$ and the pre-shock velocity perpendicular to the shock front $v_0$.

We assume a spherically symmetric stellar wind that is accelerated to its final velocity $v_{\infty}$ within a few stellar radii before any interaction takes place. For a given mass loss rate $\dot M$, the wind density at any distance $r$ from the central star is 
\begin{equation}\label{eqn:rho}
\rho(r) = \frac{\dot M}{4 \pi r^2 v_{\infty}}\ .
\end{equation}
Figure~\ref{fig:sketch} shows that $v_0$ depends on the position of the shock:
\begin{equation}
\label{eqn:v0}v_0 = v_{\infty} \sin \psi
\end{equation}
with 
\begin{equation}\label{eqn:angle}
\psi+\alpha =  \theta \ .
\end{equation}
Again, Figure~\ref{fig:sketch} shows
\begin{equation}\label{eqn:theta}
\tan\theta = \frac{\omega}{z}
\end{equation}
and the angle $\alpha$ is given by the derivative of the position of the shock front:
\begin{equation}\label{eqn:deriv}
\frac{\rm{d}\omega}{\rm{d}z} = \frac{\sin \alpha}{\cos \alpha} = \tan{\alpha}
\end{equation}
This gives:
\begin{equation}\label{eqn:psi}
\psi = \arctan{\frac{\omega}{z}} - \arctan{\frac{\rm{d}\omega}{\rm{d}z}}\ .
\end{equation}
Inserting equation~\ref{eqn:rho} and \ref{eqn:v0} into eqn.~\ref{eqn:Pofz}, we arrive at: 
\begin{equation}\label{eqn:P}
P(z) = \rho_0 v_0^2 = \frac{\dot{M}}{4\pi v_{\infty}(z^2+\omega^2)} v_{\infty}^2 \sin^2(\psi)
\end{equation}
Inserting eqn.~\ref{eqn:psi} this gives an ordinary differential equation (ODE), that describes the shape of the shock front:
\begin{equation}\label{eqn:ode}
\frac{\rm{d}\omega}{\rm{d}z} = \tan\left[\arctan\left(\frac{\omega}{z}\right)-\arcsin\left(\frac{\sqrt{z^2+\omega^2}}{R_0}\right)\right]
\end{equation}
with
\begin{equation}\label{eqn:r0}
R_0(z) = \sqrt{\frac{\dot{M} v_{\infty}}{4\pi P(z)}},
\end{equation}
where $R_0(z)$ is the maximal cylindrical radius of the shock front.

The solution to the ODE determines the location of the shock front. This allows us to calculate the pre-shock velocity perpendicular to the shock front using eqn.~\ref{eqn:v0} and the post-shock temperature $T_{\mathrm{post-shock}}$. From eqn.~\ref{eqn:RH3} with negligible pre-shock pressure \textbf{(this assumption is justified in Section~\ref{sect:T_0})} and $v_0=4\;v_1$ for a strong shock \textbf{(the assumption for a strong shock is justified in Section~\ref{sect:T_0})} we derive:
\begin{equation}
T_{\mathrm{post-shock}}(z) = \frac{3}{16} \frac{\mu m_{\textrm{H}}}{k} v_0(z)^2,\label{eqn:T}
\end{equation}
where $m_{\textrm{H}}$ denotes the mass of the hydrogen atom, $k$ the Boltzmann constant and $\mu=0.7$ the mean particle mass for a highly ionized plasma. For general $P(z)$ the ODE needs to be solved numerically\footnote{It is possible to remove all trigonometric functions from eqn.~\ref{eqn:ode} by means of addition formulae, but that introduces singularities into the solution. Thus, we numerically solve the ODE in the form of eqn.~\ref{eqn:ode}.}. This is done in an IPython notebook \citep{PER-GRA:2007}. All code is available at \url{https://github.com/hamogu/RecollimationXrayCTTS/}.
