To derive the position of the shock front in the $(z, \omega)$ plane where the pre-shock ram pressure of the stellar wind and the post-shock pressure equal the external pressure $P(z)$, we need to calculate the pre-shock density $\rho_0$ and the pre-shock velocity perpendicular to the shock front $v_0$.

We assume a pressure driven stellar wind, that is spherically symmetric and is accelerated to its final velocity $v_{\infty}$ within a few stellar radii before any interaction takes place. For the given mass loss rate $\dot M$ of the stellar wind, the wind density at any distance $r=\sqrt{z^2+\omega^2}$ from the central star is 
\begin{equation}\label{eqn:rho}
\rho(r) = \frac{\dot M}{4 \pi r^2 v_{\infty}}\ .
\end{equation}
Figure~\ref{fig:sketch} shows that $v_0$ depends on the position of the shock:
\begin{equation}
\label{eqn:v0}v_0 = v_{\infty} \sin \psi
\end{equation}
with 
\begin{equation}\label{eqn:angle}
\psi+\alpha = - \theta \ .
\end{equation}
Again, Figure~\ref{fig:sketch} shows how two of the angles in this equation can be calculated, so that $\psi$ can be determined. First,
\begin{equation}\label{eqn:theta}
\tan\theta = \frac{\omega}{z}\ ;
\end{equation}
second, the angle $\alpha$ is given by the derivative of the position of the shock front:
\begin{equation}\label{eqn:deriv}
\frac{\rm{d}\omega}{\rm{d}z} = \frac{\sin \alpha}{\cos \alpha} = \tan{\alpha}
\end{equation}
This gives:
\begin{equation}\label{eqn:psi}
\psi = \arctan{\frac{\omega}{z}} - \arctan{\frac{\rm{d}\omega}{\rm{d}z}}\ .
\end{equation}
Inserting equation~\ref{eqn:rho} and \ref{eqn:v0} into eqn.~\ref{eqn:4}, we arrive at an ordinary differential equation (ODE), that describes the functional form of the shape of the shock front:
\begin{equation}
P(z) = \rho_0v_0^2 = \frac{\dot{M}}{4\pi v_{\infty}(z^2+\omega^2)} v_{\infty}^2 \sin^2(\psi)
\end{equation}
Inserting eqn.~\ref{eqn:psi} this can be simplified to
\begin{equation}\label{eqn:ode}
\frac{\rm{d}\omega}{\rm{d}z} = \tan{\atan{\frac{\omega}{z}}-\asin{\frac{r}{R_0}}}
\end{equation}
with
\begin{equation}
R_0(z) = \sqrt{\frac{\dot{M} v_{\infty}}{4\pi P(z)}},
\end{equation}
where $R_0(z)$ is the zylindrical radius of the shock front if no further conditions are placed. In the case of young stars, however, there is another constraint. Due to the circum-stellar disk the stellar wind cannot expand freely in the $z=0$ plane; it is constrained to $R_{\textrm{shock}}(0) = R_{\textrm{inner}}$. Thus, we start the integration at $R_{\textrm{inner}} = 0.05$~AU.

For general $P(z)$ this ODE needs to be solved numerically. The negative root describes the compression shock that heats up the gas.