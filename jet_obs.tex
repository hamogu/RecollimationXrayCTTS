\subsection{Observational properties of jets from young stars}

The slowest velocities are observed in molecular lines with typical line shifts of only a few km~s$^{-1}$ \citep{2008ApJ...676..472B}. These molecular outflows have wide opening angles around 90$^{\circ}$ \citep[e.g.][]{2013A&A...557A.110S,2014A&A...564A..11A} and are presumably launched from the disk. Faster components are seen in H$\alpha$ or in optical forbidden emission lines such as [\ion{O}{1}] or [\ion{S}{2}]. \citet{2000ApJ...537L..49B} observed the jet from the CTTS \object{DG Tau} with seven long-slit exposures of \emph{HST}/STIS to resolve the kinematic structure of the jet both along and perpendicular to the jet axis. They find that the faster jet components are better collimated and propose an ``onion''-like scenario, where the fastest jet components make up the innermost layer and the surrounding layers have progressively lower velocities away from the jet axis. The fastest velocities seen in optical emission lines are typically 200-300~km~s$^{-1}$ \citep{2004Ap&SS.292..651B,2008ApJ...689.1112C,2013A&A...550L...1S}.