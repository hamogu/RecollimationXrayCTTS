\subsection{Observational properties of jets from young stars}

Outflows from CTTS consist of several layers with different flow velocities, densities, temperatures, chemical compositions and different launching positions. Conceptually, these can be thought of as conical shells, stacked into each other like the layers of an onion \citep{2000ApJ...537L..49B}. Observationally, it is not clear if there is a smooth transition between those layers or if they are separated by discontinuities. 

The slowest velocities are observed in molecular lines with typical line shifts of only a few km~s$^{-1}$ \citep{2008ApJ...676..472B}. These molecular outflows have wide opening angles around 90$^{\circ}$ \citep[e.g.][]{2013A&A...557A.110S,2014A&A...564A..11A} and are presumably launched from the disk. Faster components are seen in H$\alpha$ or in optical forbidden emission lines such as [\ion{O}{1}] or [\ion{S}{2}]. \citet{2000ApJ...537L..49B} observed the jet from the CTTS \object{DG Tau} with seven long-slit exposures of \emph{HST}/STIS to resolve the kinematic structure of the jet both along and perpendicular to the jet axis. They find that the inner, most collimated jet component moves fastest and surrounding layers have progressively lower velocities away from the jet axis. The fastest velocities seen in optical emission lines are typically 200-300~km~s$^{-1}$ \citep{2004Ap&SS.292..651B,2008ApJ...689.1112C,2013A&A...550L...1S}.

At some distance from the star, shock fronts, the so called Herbig-Haro (HH) objects, are observed when the jet runs into the ambient medium or when material emitted at higher velocities catches up with previously emitted slower material. \citet{2006A&A...456..189P} studied emission line ratios in jets in the Orion and Vela star forming regions. In their sample the knots are typically a few hundred AU from the central star. They find electron densities around $n_e \approx 50-300 \textrm{\;cm}^{-3}$ when looking at optical emission lines, slightly higher values for forbidden emission lines like [\ion{Fe}{2}] and $n_e \approx 5\times10^5-5\times10^6 \textrm{\;cm}^{-3}$ from Ca lines. The ionization fraction in HH objects is low, so the actual particle number density is one to two orders of magnitude higher. \citet{2004ApJ...609..261H} observed the CTTS \object{HN Tau} and resolved the jet at only 30~AU, much closer to the star. They find densities around $n_e=10^6-10^7\textrm{\;cm}^{-3}$ from [\ion{Fe}{2}] lines. This shows that the inner layers of jets can have densities significantly above those of the interstellar medium.