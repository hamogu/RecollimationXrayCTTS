\section{Discussion}
\label{sect:discussion}
We show that stellar wind and disk wind can interact in a CTTS system. The magnetic and thermodynamic pressure of the disk wind can confine the stellar wind into a narrow, jet-like region, bound by an elongated shock surface. For reasonable parameters of $\dot M$, $v_\infty$, $\omega_0$ and $P(z)$ the shock surface encloses a region a few AU wide and tens of AU along the jet axis, but our model makes no statement about emission that originates further out in the jet such as Herbig-Haro knots. Only a fraction of the kinetic energy of the stellar wind is converted into heat in the recollimation shock and the remaining velocity can still be sufficient to heat the jet material again when it encounters another obstacle, such as the ISM.

Most of the imaging of YSO winds traces molecular lines and low-ionization stages, e.g. \ion{O}{1} or \ion{Fe}{2}. These lines are formed in low-temperature regions, but not in a hot post-shock plasma. Thus, one could expect to see a hole that is filled by hot post-shock plasma from the stellar wind. However, no such hole is resolved in any CTTS imaging. Our calculations show that the shock surface is so small that it cannot be resolved with current instrumentation\footnote{HST imaging and AO corrected, ground-based IR observations reach a resolution around 0\farcs1, which corresponds to 15~AU for DG Tau -- \textbf{one of }the closest YSO jets. However, saturation or coverage by a coronagraphic disk often mean that even structures slightly larger can be missed in images, if they are located very close to the central star.} and therefore cannot be seen directly as a cavity in the disk wind. A small fraction of the stellar wind is shocked to X-ray emitting temperatures $>1$~MK and provides a stationary X-ray source consistent with observations. 
We show that such a shock naturally arises in a scenario where the stellar wind feeds the innermost layer of the jet because it is confined by external pressure.
Furthermore, \citet{2013A&A...550L...1S} observed C\;{\sc iv} emission in DG Tau that is formed at cooler temperatures than those required for X-ray emission. In recent observations in the IR \citet{2014arXiv1404.0728W} also identified a stationary emission region on the jet axis about 40~AU from the central star. They interpret the X-rays,  C\;{\sc iv}, and their own [ Fe\;{\sc ii}] data all as a signature of the same shocked jet, while \citet{2013A&A...550L...1S} point out that the C\;{\sc iv} luminosity is too large to be powered by just the cooling X-ray plasma. Looking at the post-shock temperature distribution in Fig.~\ref{fig:result}, our model can naturally explain multiple temperature components in the stellar wind.

However, more detailed numerical simulations of the post-shock cooling zone and the shape of the contact discontinuity between the disk wind and the post-shock stellar wind are required to check whether the physical extent of the cooling region behind the shock front and the position of the peak  C\;{\sc iv} emission can be matched to the observations.

In any case, Figure~\ref{fig:fit} shows that the model can explain the observed X-ray spectra. The best-fit values obtained for DG~Tau have a significantly higher velocity and lower mass loss rate than the fiducial model. The parameters for the fiducial model are chosen to match the flow velocities and mass loss rates that are observed in jets. Yet, the higher velocity and lower mass loss rate do not directly contradict those observations, because the pre-shock stellar wind extends only over a small area, so that it presumably contributes little to the luminosity in the optical emission lines compared to the inner disk wind. Consequently, the high velocity in the stellar wind is not directly observable. We also note that in the fit we varied only the normalization of the external pressure, but not the spatial profile. Different profiles lead to different shock-front shapes and shock velocities and thus require an adjustment of $v_\infty$ and $\dot M$ to fit the X-ray data. However, the pressure profile caused by the disk wind and disk magnetic field is not very well constrained. Thus, the accuracy of the fitted numbers is not so much limited by the statistical uncertainty given in table~\ref{tab:fiducial}, but by the systematics of the model. The best-fit values should not be taken at face value, but they demonstrate that a recollimation shock is one possible explanation for the observed X-ray emission. 

The idea of a recollimation shock is not new.
\citet{1993ApJ...409..748G} discussed a similar idea as we do here, where they aim to explain the forbidden optical emission lines seen from CTTS with a shock due to the recollimation of the jet outflow. In contrast to our model, they attribute it to the shocked disk wind, not the stellar wind. However, a shocked disk wind cannot supply the high shock velocities to explain the resolved X-ray and \ion{C}{4} emission that we now see. Our model, a shocked stellar wind, is collimated because it is embedded into a strong disk wind. We expect that the low-temperature emission from the stellar wind is small compared to the low-temperature emission from the surrounding disk wind. Only for high temperatures (X-ray and FUV emission), the stellar wind will dominate because it is much faster.

\textbf{We now compare our work} to the simulations of \citet{2010A&A...511A..42B,2010A&A...517A..68B} \textbf{which use smooth velocity and density profiles for the jet with radii between 8 and 200~AU -- larger than almost all simulations shown in this article.  \citet{2010A&A...517A..68B} find X-ray emission of varying luminosity around 100~AU from the central source in their simulations for the HH~154 jet and it is very likely that slightly different jet parameters can cause this feature to appear closer to the star. However, this emission is much weaker than the X-ray emission at larger scales in contrast to the situation in DG~Tau. Further simulations are required to test if realistic launching conditions can also make a quasi-stationary X-ray shock that outshines the knots at larger distances. Another possibility that can be tested in future simulations is that our picture of a wind-wind interaction can be combined with a time variable launching speed. At distances of only a few AU the stellar wind and the disk wind would interact and cause a stationary collimation shock as explained in this article. After passing through the shock, the stellar wind and the outer disk wind might mix, so that the jet could appear more homogeneous at larger distances. If the intital launching velocity is time variable, not only will the properties of the recollimation shock change, but the velocity of the combined outflow would also vary and could thus cause moving shock fronts further out in the jet as in the simulations of \citet{2010A&A...511A..42B,2010A&A...517A..68B}.
\citet{2011ApJ...737...54B} find a diamond-shaped shock and again it is very likely that slightly different jet parameters can cause this feature to appear closer to the star as seen in DG~Tau. }

\citet{2009A&A...502..217M,2012A&A...545A..53M} also perform numerical simulations of a jet confined by a disk wind. Their simulations again deal with larger distances from the central star and they concentrate on knots in the jet. Yet, their bubbles of shock heated gas have very similar shapes compared with our results in Figure~\ref{fig:result}. This indicates that this form is robust. 

Does the scenario of a stellar wind recollimation shock as X-ray source also apply to other CTTS or is it specific to DG Tau? While there is no reason to believe that the wind launching in DG~Tau is unique, it certainly presents us with a special viewing geometry, where the star itself is heavily absorbed, but the jet shock at 30~AU is visible. At the same time, the first knot in the jet that shows X-ray emission is located at 700~AU and thus can be clearly separated from the inner, presumably stationary emission. \citet{2011A&A...530A.123S} analyze three epochs of X-ray emission from HH~154 and see an inner stationary component and a variable component at slightly larger radii \textbf{\citep[see also][]{2011ApJ...737...54B}}. However, there is no gap between both components and the small number of photons makes it difficult to quantify variability and proper motion. Also, the emission from HH~154 is more energetic and would require much larger wind velocities if it is due to a stellar wind recollimation shock. In other young stars with resolved X-ray emission, the central star is either visible and outshines any potential recollimation shock (e.g.\ HD~163296 \citep{2005ApJ...628..811S,2013A&A...552A.142G} or RY~Tau \citep{2014ApJ...788..101S}) or embedded so deep into the cloud that a wind shock at a few tens of AU would be completely absorbed \citep[e.g.\ HH80/81][]{2004ApJ...605..259P}. Thus, we cannot decide this question observationally, but it seems reasonable to assume that the same processes that lead to a recollimation shock in the stellar wind in DG~Tau, should also operate in other CTTS, even if less favourable conditions make it harder to observe in X-rays.

