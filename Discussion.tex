\section{Discussion}
\label{sect:discussion}
In a CTTS system stellar wind and disk wind interact. The high pressure of the disk wind can confine the stellar wind into a narrow, jet-like region, bound by an elongated shock surface. For reasonable parameters of $\dot M$, $v_\infty$, $\omega_0$ and $P(z)$ the shock surface encloses a region only several AU wide but tens of AU along the jet axis. 
 
The shock surface is so small that it cannot be resolved with current instrumentation and therefore cannot be seen directly as a cavity in the disk wind. A small fraction of the stellar wind is shocked to X-ray emitting temperatures $>1$~MK and provides a stationary X-ray source consistent with observations. 
Paper~I showed that a shock with these properties is required to explain the observed X-ray emission in the CTTS DG~Tau if shocks are major heating agents. We show that such a shock naturally arises in a scenario where the stellar wind is confined by an external pressure and feeds the innermost layers of the jet.
Furthermore, \citet{2013A&A...550L...1S} observed C\;{\sc iv} emission in DG Tau that is formed at cooler temperatures than those required for X-ray emission and that is too luminous to be explained by cooled X-ray plasma alone. Almost all solutions of the ODE describing the interaction of stellar and disk wind have more plasma heated to 0.5~MK than to 1~MK and can in principle explain these observations as well. In recent observations in the IR \citet{2014arXiv1404.0728W} also identified a stationary emission region on the jet axis about 40~AU from the central star. They interpret the X-rays,  C\;{\sc iv}, and their own [ Fe\;{\sc ii}] data all as a signature of the same shocked jet, while \citet{2013A&A...550L...1S} point out that the C\;{\sc iv} luminosity is too large to be powered by just the cooling X-ray plasma. Looking at the post-shock temperature distribution in Fig.~\ref{fig:result}, our model can naturally explain how multiple temperature components arise in the stellar wind.

However, more detailed numerical simulations of the post-shock cooling zone and the shape of the contact discontinuity between the disk wind and the post-shock stellar wind are required to check whether the physical extent of the cooling region behind the shock front and the position of the peak  C\;{\sc iv} emission can be matched to the observations.

\citetp{1993ApJ...409..748G} discussed a similar idea as we do here, where they aim to explain the forbidden optical emission lines seen from CTTS with a shock due to the recollimation of the jet outflow. In contrast to our model, they attribute it to the shocked disk wind, not the stellar wind. At the time, the high-temperature emission from CTTS jets was not known and that seems to be a natural choice to keep the shock velocity low. However, a shocked disk wind cannot supply the high shock velocities to explain the resolved X-ray and \ion{C}{4} emission that we now see. On the other hand, it is possible that our model, a shocked stellar wind, will also produce some fraction of the optical line emission.

There is too much parameter degeneracy to turn the argument around and derive $\dot M$ and  $v_\infty$ from the fact that we observe X-ray and FUV emission a few tens of AUs from the star, but our models requires a certain range of external pressures $P(z)$. Too high pressures push the shock front back to the jet axis very close to the star and too low pressures cannot confine the shock region within a few tens of AU.

