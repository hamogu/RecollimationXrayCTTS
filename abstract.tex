Emission from jets from pre-main-sequence stars is mostly observed in the optical and IR wavelength range. However, in several cases X-ray and UV observations reveal a weak but highly energetic component in those jets. If this component is heated in shocks then the required velocities are of the order 300-500~km~s$^{-1}$ for standing shocks and higher for moving shock fronts. In this article we show semi-analytically that a recollimation boundary layer between a fast stellar wind and a slower, more massive disk wind can have the right properties to explain the observed X-ray emission, provided that the stellar wind is very massive. Our calculations support a wind-wind interaction scenario for the high energy emission near the base of YSO jets. 
