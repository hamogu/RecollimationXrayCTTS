Young stars accrete mass fomr circumstellar disks. In many cases, the accretions coincides with a phase of massive outflows, which can be highly collimated. Those jets from pre-main-sequence stars emit predominantly in the optical and IR wavelength range. However, in several cases X-ray and UV observations reveal a weak but highly energetic component in those jets. X-rays are observed both from stationary regions close to the star and from knots in the jet several hundred AU from the star. 
In this article we show semi-analytically that a fast, stellar wind which is recollimated by the pressure from a slower, more massive disk wind can have the right properties to power stationary X-ray emission. The size of the shock regions is compatible with the observational constraints. Our calculations support a wind-wind interaction scenario for the high energy emission near the base of YSO jets. For the specific case of DG~Tau, a stellar wind with a mass loss rate of $5\cdot10^{-10}\;M_{\odot}\mathrm{ yr}^{-1}$ and a wind speed of 800~km~s$^{-1}$ reproduces the observed X-ray spectrum.
We conclude that stellar wind recollimation shocks are one possible scenario to power stationary X-ray emission close to the jet launching point.
