All jets from pre-main-sequence stars emit in the optical and IR wavelength range. However, in several cases X-ray and UV observations reveal a weak but highly energetic component in those jets. If this component is heated in shocks then the required velocities are $>$300~km~s$^{-1}$ for standing shocks and higher for moving shock fronts. In this article we show semi-analytically that a fast, stellar wind which is recollimated by the pressure from a slower, more massive disk wind can have the right properties to explain the observed X-ray emission. The size of the shock regions is compatible with the observational contraints. Our calculations support a wind-wind interaction scenario for the high energy emission near the base of YSO jets. 
For the specific case of DG~Tau, a stellar wind with a mass loss rate of $5\cdot10^{-10}\;M_{\odot}\mathrm{ yr}^{-1}$ and a wind speed of 800~km~s$^{-1}$ reproduces the observed X-ray spectrum.
