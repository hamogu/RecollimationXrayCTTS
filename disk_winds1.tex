\section{Disk winds as boundary conditions for stellar winds}
Different models exist to explain wind launching from the stellar surface \citep{1988ApJ...332L..41K,2005ApJ...632L.135M}, the X-point close to the inner disk edge \citep{1994ApJ...429..781S} and magneto-centrifugal launching from the disk \citep{1982MNRAS.199..883B,2005ApJ...630..945A}. It is likely that more than one mechanism contributes to the total outflow from the system. In this case, we expect a contact discontinuity between the different components. Numerically, the magneto-centrifugally accelerated disk wind is probably the best explored component. Magneto-hydrodynamic (MHD) simulations of the disk wind have been performed in 2D \citep{2005ApJ...630..945A}, 2.5D \citep{2011ApJ...728L..11R} or 3D \citep{2006ApJ...653L..33A}, but typically do not resolve the stellar wind, where the magneto-centrifugal launching is not effective. However, they show that the disk wind is collimated close to the axis and that the densities are largest in this region. Furthermore, the inner layers of the outflow close to the jet are within the Alfv\'en surface, the boundary between a magnetically dominated flow and a gas-pressure dominated flow, even at distances of several tens of AU form the central star, in contrast to the outer, less collimated layers of the wind, which leave the magnetically dominated region at a few AU.

http://adsabs.harvard.edu/abs/2008A%26A...491..339S

http://adsabs.harvard.edu/abs/2009A%26A...502..217M

http://adsabs.harvard.edu/abs/2010A%26A...516A...6S
compared this to Bacciotti type HST data, but since X-ray shock not visible there that does not really help us for the context of this article.