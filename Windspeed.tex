\subsection{Wind speed}
The launching mechanism of the stellar wind in CTTS is uncertain. \citet{2007IAUS..243..299M} show that stellar winds from CTTS cannot have a total mass loss above $10^{-11}M_\odot\mathrm{ yr}^{-1}$ if they are launched hot. 
Thus, the winds of CTTS are probably more complex than just a scaled up version of the solar wind.
Still, \textbf{launching velocities similar to the solar wind seem to be a reasonable estimate for $v_\infty$. A velocity of a few hundred km~s$^{-1}$ is compatible with the fastest speeds observed from optical line shifts in jets (see references in Section~\ref{sect:introjetobs}). It also matches the depth of the graviotational potential, and while the launching mechanism is unknown it is likely powered by energy released in the accretion process \citep{1988ApJ...332L..41K,2005ApJ...632L.135M}.}

The solar wind consists of a slow wind with a typical velocity of 400~km~s$^{-1}$ and a fast wind around 750~km~s$^{-1}$ \citep{2005JGRA..110.7109F}. The relative contribution and the launching position of the two types changes over the solar cycle, but the slow wind often emerges from regions near the solar equator and the fast wind is generally associated with coronal holes \citep{1999GeoRL..26.2901G,2003A&A...408.1165B,2009LRSP....6....3C}. Despite these differences, the total energy flux in the solar wind is almost independent of the latitude, because the slower wind is denser than the faster wind \citep{2012SoPh..279..197L}. We set $v_\infty=600$~km~s$^{-1}$ as the fiducial outflow velocity and we assume that the wind is accelerated close to the star and has reached $v_\infty$ before it interacts with the shock front. We use a spherically symmetric stellar wind with a constant velocity. For a solar-type wind this works well for deriving the shape of the shock front because eqn.~\ref{eqn:r0} depends only on the total energy flux $\rho v^2_\infty \propto \dot M v_\infty$ and not the velocity itself. 

Figure~\ref{fig:v_infty} shows how a large $v_\infty$ and a correspondingly large ram pressure push the shock front higher above the disk plane, similar to outflows with a larger $\dot M$. Additionally, $v_\infty$ is the most important parameter that controls the post-shock temperatures.
The figure shows that high shock speeds and thus high post-shock temperatures are reached close to the disk plane. Because this region covers a large solid angle, it is an important contributor to the total temperature distribution of the post-shock plasma (rightmost panel). However, the temperature in this region is probably overestimated by our model because the wind may not have reached $v_\infty$ this close to the star or the wind may not be spherically symmetric. A lower wind velocity and higher density at the equator similar to the solar low-velocity wind would lead to lower post-shock temperatures with a higher emission measure close to the disk plane.
