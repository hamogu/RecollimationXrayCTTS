\subsection{Wind speed}
The launching mechanism of the stellar wind in CTTS is uncertain. For a solar-type wind, it is not unreasonable to expect similar wind speeds. The solar wind consists of a slow wind with a typical velocity of 400~km~s$^{-1}$ and a fast wind around 750~km~s$^{-1}$ \citep{2005JGRA..110.7109F}. The relative contribution and the launching position of the two types changes over the solar cycle, but the slow wind often emerges from regions near the solar equator and fast wind is generally associated with coronal holes \citep{1999GeoRL..26.2901G,2003A&A...408.1165B,2009LRSP....6....3C}. Despite these differences, the total energy flux in the solar wind is almost independent of the latitude, because the slower wind is denser than the faster wind \citep{2012SoPh..279..197L}. In this article, we use $v_\infty=500$~km~s$^{-1}$ as the fiducial outflow velocity and we assume that the wind is accelerated close to the star and has reached $v_\infty$ before it interacts with the shock front.

In our model, we use a spherically symmetric stellar wind with a constant velocity. For a solar-type wind this works well for deriving the shape of the shock front because eqn.~\ref{eqn:r0} contains the kinetic energy density $\rho v^2_\infty \propto \dot M v_\infty$. However, a lower wind velocity and higher density at the equator would lead to lower post-shock temperatures with a higher emission measure close to the disk plane. ({\bf Hans: this last sentence does not connect well with the sentence before it.})

\citet{2007IAUS..243..299M} show that stellar winds from CTTS cannot have a total mass loss above $10^{-11}M_\odot\mathrm{ yr}^{-1}$ if they are launched hot. Otherwise, the high densities required to reach such a mass loss would lead to a runaway cooling. This should be observable and would probably hinder the wind  launching. Thus, the winds of CTTS are probably more complex than just a scaled up version of the solar wind. Still, the wind speeds observed in the sun provide a reasonable estimate for $v_\infty$.

Figure~\ref{fig:v_infty} shows how a large $v_\infty$ and a correspondingly large ram pressure of the stellar wind push the shock front higher above the disk plane, similar to outflows with a larger $\dot M$. Additionally, $v_\infty$ is the single most important parameter that controls the maximal post-shock temperatures and the amount of hot plasma.