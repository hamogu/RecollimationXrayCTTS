\subsection{X-ray emission from stellar jets}
\label{sect:introxray}
In some jets there is evidence for another component that is faster than what is seen in optical lines and that radiates at higher energies. X-rays were first seen in the jet \object{HH 2} \citep{http://adsabs.harvard.edu/abs/2001Natur.413..708P,http://adsabs.harvard.edu/abs/2012A%26A...542A.123S}, where the central star is invisible. Later, X-ray emission was also discovered from CTTS with are less embedded, most notebly \object{DG Tau} (see below) and \object{RY Tau}\citep{http://adsabs.harvard.edu/abs/2014ApJ...788..101S}. DG~Tau is the best case to study X-ray emission from the jet close to the star for two reasons: (i)~No other jet driving young stellar object has been observed as often in X-rays as DG~Tau. It was the target of several shorter \emph{Chandra} exposures in 2004, 2005, and 2006 and a large program in 2010 \citep{2005ApJ...626L..53G,2008A&A...478..797G,2011ASPC..448..617G} and has been observed with \emph{XMM-Newton} in 2004 \citep{http://adsabs.harvard.edu/abs/2007A&A...468..353G} and in 2012 \citep{SchneiderDGTauXray}. (ii)~DG~Tau itself is hidden behind a column density of \textbf{AandA is down, I'll check that number later.} \citep{http://adsabs.harvard.edu/abs/2008A%26A...478..797G}, which absorbs any soft X-ray emission from the central star. Hard, coronal emission pierces through the gas and allows us to pinpoint the stellar position to high accuracy, while the soft X-rays observed close to the stellar position must come from the jet. In contrast, in RY Tau and \object{HD 163296} \citep{http://adsabs.harvard.edu/abs/2013A%26A...552A.142G}, the stellar soft emission outshines any potential jet emission close to the central star.

Due to this unique situation we can distinguish three different regions of X-ray emission in DG~Tau: First, weak and soft emission from the jet is resolved several hundred AU from the star itself. Second, hard emission from the central star is observed with stellar flares as seen on many other young and active stars. Third, soft X-rays are emitted in a region about 30-40~AU above the plane of the accretion disk. The centroid of the spatial distribution of soft X-rays is consistent with a position on the jet axis 30-40~AU from the star \citep{2008A&A...488L..13S}. The temperature of this inner emission region is remarkably stable over one decade with value between 3 and 4~MK. The maximum change observed is about 25\,\% \citep{SchneiderDGTauXray}.

There is no reason to believe that the DG~Tau system represents exceptional conditions for jet launching; on the contrary, in this article we regard it as a proto-type for an X-ray emitting jet. While the inclination and absorption are less favourable to observe X-ray emission in the jet very close to the star for other CTTS system, similar indications are found in  \object{HH 154}, which also shows an inner, stationary X-ray component and additional emission in the knots \citep{2010A&A...511A..42B,2011A&A...530A.123S} and in the more massive Herbig Ae/Be star \object{HD 163296} there are indications that the X-ray emission is extended in the direction of the jet by a few dozen AU, too \citep{2005ApJ...628..811S,2013A&A...552A.142G}.

In \citet{2009A&A...493..579G} (from now on ``paper I'') we showed that the soft X-ray emission close to DG~Tau can be explained by shock heating of a jet component moving with 400-500~km~s$^{-1}$. The mass flux in this component is less than $10^{-3}$ of the total mass flux in the jet or even lower if the same material is reheated in several consecutive shocks. If the density in the fast outflow is $>10^5$~cm$^{-3}$ then the cooling length of this shock is only a few AU and would be invisible in current optical and IR observations, since it would be surrounded and outshined by the more luminous emission from the more massive, but slower jet component. However, the stationary nature of the X-ray emission remained unexplained in the scenario discussed in paper~I.