\subsection{X-ray emission from stellar jets}
\label{sect:introxray}
In some jets there is evidence for another, hotter and faster component. X-rays were first seen in the jet \object{HH 2} \citep{2001Natur.413..708P,2012A&A...542A.123S}, where the central star is invisible. Later, X-ray emission was also discovered from less enbedded CTTS, \object{RY Tau} \citep{2014ApJ...788..101S} and, most notably, \object{DG Tau}. DG~Tau is the best case to study X-ray emission from the jet close to the star for two reasons: (i)~No other jet driving young stellar object has been observed as often in X-rays. DG~Tau was the target of several shorter \emph{Chandra} exposures in 2004, 2005, and 2006 and a large program in 2010 \citep{2005ApJ...626L..53G,2008A&A...478..797G,2011ASPC..448..617G} and has been observed with \emph{XMM-Newton} in 2004 \citep{2007A&A...468..353G} and in 2012 \citep{SchneiderDGTauXray}. (ii)~DG~Tau itself is hidden behind a column density of $N_{\textrm{H}}=2\times10^{22}\textrm{ cm}^{-3}$ \citep{2008A&A...478..797G}, which absorbs any soft X-ray emission from the central star. Hard, coronal emission pierces through the gas and allows us to pinpoint the stellar position to high accuracy, while the soft X-rays observed close to the stellar position must come from the jet.

We can distinguish three different X-ray emitting regions in the DG~Tau system: First, hard emission from the central star is observed with stellar flares as seen on many other young and active stars. Second, weak and soft emission from the jet is resolved several hundred AU from the star itself. Third, additional soft X-rays are emitted close to, but not from the star, because they are subject to a much smaller absorbing column density than the central, coronal source. The centroid of the spatial distribution of soft X-rays is consistent with a position on the jet axis 30-40~AU from the star \citep{2008A&A...488L..13S,2011ASPC..448..617G}. The temperature of this inner emission region is remarkably stable over one decade between 3 and 4~MK; the maximum change observed is about 25\,\%. The change in luminosity is 1.6 in the same time range ($L_X=1-2\times10^{30}\textrm{ erg s}^{-1}$) \citep{SchneiderDGTauXray}.

There is no reason to believe that the DG~Tau system represents exceptional physical conditions for jet launching. While the inclination and absorption are less favorable to observe X-ray emission very close to the star for other CTTS systems, there are indications that \object{HH 154} also shows an inner, stationary X-ray component and additional emission in the knots \citep{2010A&A...511A..42B,2011A&A...530A.123S} and that the X-ray emission in the more massive Herbig Ae/Be star \object{HD 163296} is extended in the direction of the jet by a few dozen AU, too \citep{2005ApJ...628..811S,2009A&A...494.1041G,2013A&A...552A.142G}.

In \citet{2009A&A...493..579G} we showed that the soft X-ray emission close to DG~Tau can be explained by shock heating of a jet component moving with 400-500~km~s$^{-1}$. The mass flux in this component is less than $10^{-3}$ of the total mass flux in the jet or even lower if the same material is reheated in several consecutive shocks. If the density in the fast outflow is $>10^5$~cm$^{-3}$ then the cooling length of this shock is only a few AU and would be invisible in current optical and IR observations, since it would be surrounded and outshone by the more luminous emission from the more massive, but slower jet component. However, the stationary nature of the X-ray emission remained unexplained in that article.