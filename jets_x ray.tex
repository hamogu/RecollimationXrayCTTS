\subsection{X-ray emission from stellar jets}

Yet, in some jets from CTTS there is evidence for another, more energetic, component. The best studied case is DG~Tau that was the target of several shorter \emph{Chandra} exposures in 2004, 2005, and 2006 and a large program in 2010 \citep{2005ApJ...626L..53G,2008A&A...478..797G,2011ASPC..448..617G}. These observations showed X-ray emission from three distinct regions: First, weak and soft emission from the jet is resolved several hundred AU from the star itself. Second, hard emission from the central star is observed with stellar flares as seen on many other young and active stars. Since the star itself is embedded in circumstellar material, the stellar soft X-ray emission is expected to be completely absorbed. However, soft X-rays very close to the star are observed; they are emitted in a region about 30-40~AU above the plane of the accretion disk. The centroid of the spatial distribution of soft X-rays is consistent with a position on the jet axis 30-40~AU from the star, but the uncertainties on the position would also allow an off-axis emission region \citep{2008A&A...488L..13S}. The luminosity and temperature of this inner emission region are remarkably stable over one decade. The maximum change observed is about 25\,\% \citep{SchneiderDGTauXray}.

DG~Tau is the best observed case, but a similar scenario probably applies to other jet launching young stars, e.g.\ \object{HH 154} also shows an inner, stationary X-ray component and additional emission in the knots \citep{2010A&A...511A..42B,2011A&A...530A.123S}.
In the more massive Herbig Ae/Be star \object{HD 163296} there are indications that the X-ray emission is extended in the direction of the jet by a few dozen AU, too \citep{2005ApJ...628..811S,2013A&A...552A.142G}.

In \citet{2009A&A...493..579G} (from now on ``paper I'') we showed that this inner X-ray emission can be explained by shock heating of a jet component moving with 400-500~km~s$^{-1}$. For the case of DG~Tau the mass flux in this component is less than $10^{-3}$ of the total mass flux in the jet or even lower if the same material is reheated in several consecutive shocks. If the density in the fast outflow is $>10^5$~cm$^{-3}$ then the cooling length of this shock is only a few AU and in the optical it would be unresolved and outshined by the more luminous emission from the more massive, but slower jet component. However, the stationary nature of the X-ray emission remained unexplained in this scenario.