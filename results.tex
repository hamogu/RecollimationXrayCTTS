\section{Model}
Different models exist to explain wind launching from the stellar surface \citep{1988ApJ...332L..41K,2005ApJ...632L.135M}, the X-point close to the inner disk edge \citep{1994ApJ...429..781S} and magneto-centrifugal launching from the disk \citep{1982MNRAS.199..883B,2005ApJ...630..945A}. It is likely that more than one mechanism contributes to the total outflow from the system. In this case, we expect a contact discontinuity between the different components. Numerically, the magneto-centrifugally accelerated disk wind is probably the best explored component. Magneto-hydrodynamic (MHD) simulations of the disk wind have been performed in 2D \citep{2005ApJ...630..945A} or 3D \citep{2006ApJ...653L..33A}, but typically do not resolve the stellar wind, where the magneto-centrifugal launching is not effective. However, they show that the disk wind is collimated close to the axis and that the densities are largest in this region. Furthermore, the Alfv\'en surface, 
