\subsection{Mass-loss rates}
\label{sect:masslossrates}
The measured mass loss rates in the outflows from CTTS vary widely between objects.  Even for a single object, very different mass loss rates can be found, depending on the spectral tracers chosen and on the assumptions used to calculate mass loss rates from line fluxes. The filling factor that describes the fraction of the observed volume occupied by hot gas is especially uncertain because the innermost jet component is generally not resolved.

Typical mass loss rates found in the literature for CTTS outflows are in the range $10^{-10}-10^{-6}M_{\odot}\textrm{ yr}^{-1}$ \citep{1999A&A...342..717B,2006A&A...456..189P}. \citet{2006ApJ...646..319E} measure values down to $10^{-10}$~M$_{\odot}$~yr$^{-1}$ for some CTTS, but only upper limits for weak-line T Tauri stars (WTTS). In the specific case of DG~Tau \citet{1997A&A...327..671L} calculate the  mass loss rate as $6.5\cdot 10^{-6}$~M$_{\odot}$~yr$^{-1}$; \citet{1995ApJ...452..736H}
obtain $3\cdot 10^{-7}$~M$_{\odot}$~yr$^{-1}$ and, further out in the jet, \citet{2000A&A...356L..41L} find $1.4\cdot 10^{-8}$~M$_{\odot}$~yr$^{-1}$. Those measurements for the optical jet are probably dominated by the disk wind \citep[e.g.][]{2014arXiv1404.0728W} and unlikely to track the stellar mass loss correctly.
\citet{2009A&A...493..579G} shows that a mass loss below $10^{-10}$~M$_{\odot}$~yr$^{-1}$ is sufficient to explain the X-ray emission from the jet as shock heating.
We use $10^{-8}$~M$_{\odot}$~yr$^{-1}$ as fiducial stellar mass loss in the remainder of the article. This is only a fraction to the total mass loss of the system because the disk wind, though slower, operates over a much larger area dominates the system's mass loss in our scenario.

Figure~\ref{fig:dot_m} shows how a larger mass loss rate and therefore a higher density and ram pressure in the stellar wind pushes the shock front out to larger radii and heights. The different shape of the shock front also influences the post-shock temperatures. In the high mass loss rate scenario (black dotted line) the shock front reaches its maximum radius at 60~AU and most of the spherically symmetric wind passes the shock front at shallow angles, so this scenario has the highest fraction of low temperature material.
