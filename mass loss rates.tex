\subsection{Mass-loss rates}
The measured mass loss rates in the outflows from CTTS vary widely between objetcs and even for a single objects very different mass loss rates can be found, depending on the spectral tracers chosen and on the assumptions required to calculate mass loss rates from lines fluxes. One notoriously uncertain variable is the filling factor that describes which fraction of the observed volume is occupied by the hot gas. The innermost, fastest jet component is generally not resolved, so that this question cannot be answered observationally. Also, measurement of the mass loss in a jet are only possible once the outflow can be seen over the much brigther emission of the central star. 

Typical mass loss rates found in the literature for CTTS outflows are in the range $10^{-8}-10^{-6}M_{\odot}\textrm{ yr}{-1}$ \citep{1999A&A...342..717B,2006A&A...456..189P}. In the specific case of the well-studied jet from DG~Tau \citet{1997A&A...327..671L} calculate the  mass loss rate as $6.5\cdot 10^{-6}$~M$_{\odot}$~yr$^{-1}$; \citet{1995ApJ...452..736H}
obtain $3\cdot 10^{-7}$~M$_{\odot}$~yr$^{-1}$ and a further out in the jet \citet{2000A&A...356L..41L} find $1.4\cdot 10^{-8}$~M$_{\odot}$~yr$^{-1}$. 
\citet{2009A&A...493..579G} show that only a small mass loss rate is required to explain the X-ray emission from the jet as shock heating, and it is possible that the optical jet further out entrains some disk wind material, so it might not track the stellar mass loss correctly.
Therefore, we conservatively use $1.4\cdot 10^{-8}$~M$_{\odot}$~yr$^{-1}$, a value on the low end of the suggested mass loss rates, as fiducial stellar mass loss in the remainder of the article.