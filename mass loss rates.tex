\subsection{Mass-loss rates}
The measured mass loss rates in the outflows from CTTS vary widely between objects.  Even for a single object, very different mass loss rates can be found, depending on the spectral tracers chosen and on the assumptions used to calculate mass loss rates from line fluxes. The filling factor that describes the fraction of the observed volume occupied by hot gas is especially uncertain because the innermost jet component is generally not resolved. Also, measurements of the mass loss in a jet are possible only where the jet emission is not dominated by that of the much brighter central star. 

Typical mass loss rates found in the literature for CTTS outflows are in the range $10^{-8}-10^{-6}M_{\odot}\textrm{ yr}{-1}$ \citep{1999A&A...342..717B,2006A&A...456..189P}. In the specific case of the well-studied jet from DG~Tau \citet{1997A&A...327..671L} calculate the  mass loss rate as $6.5\cdot 10^{-6}$~M$_{\odot}$~yr$^{-1}$; \citet{1995ApJ...452..736H}
obtain $3\cdot 10^{-7}$~M$_{\odot}$~yr$^{-1}$ and, further out in the jet, \citet{2000A&A...356L..41L} find $1.4\cdot 10^{-8}$~M$_{\odot}$~yr$^{-1}$. 
\citet{2009A&A...493..579G} show that only a small mass loss rate is required to explain the X-ray emission from the jet as shock heating, and it is possible that the optical jet further out entrains some disk wind material, so it might not track the stellar mass loss correctly.
Therefore, we conservatively use $1.4\cdot 10^{-8}$~M$_{\odot}$~yr$^{-1}$, a value on the low end of the suggested mass loss rates, as fiducial stellar mass loss in the remainder of the article.

Figure~\ref{fig:dot_m} shows how a larger mass loss rate and therefore a higher density and ram pressure in the stellar wind pushes the shock front out to larger radii and heights. The different shape of the shock front also influences the post-shock temperatures. In the high mass loss rate scenario (black dotted line) the shock front reaches its maximum radius at 60~AU and most of the spherically symmetric wind passes the shock front at shallow angles, so this scenario has the highest fraction of low temperature material.
